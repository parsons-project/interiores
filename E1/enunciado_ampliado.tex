\subsubsection{Objetivo}
`Diseño de interiores' es un software destinado al diseño de habitaciones
y distribución de su mobiliario. La aplicación es capaz de generar diseños
de habitaciones cumpliendo con una serie de restricciones, permitiendo
al usuario definir cuáles y cómo deben ser estas restricciones.

\subsubsection{Flujo de la aplicación}
En este apartado se explica cuál es el flujo de la aplicación, es decir,
cuáles son las posibilidades del usuario desde que abre el programa hasta que
resuelve su problema. En los siguientes apartados se explica con mucho más
detalle cada uno de los elementos más importantes que conforman el software.

\newpar
Al iniciar la aplicación se le pregunta al usuario si desea:
\begin{description}
  \item[Crear un nuevo diseño]
    En este caso el usuario deberá especificar el tipo de habitación que
    quiere crear y con qué dimensiones.

  \item[Abrir un diseño ya existente]
    El usuario podrá abrir diseños que haya guardado previamente. En este caso
    se cargará la habitación en el mismo estado que cuando se guardó para que
    el usuario pueda seguir trabajando en su diseño.
\end{description}

Una vez el usuario haya creado o abierto una habitación, la aplicación pasará
a mostrarle diversos elementos para facilitarle el diseño de la habitación.
Estos son:

\begin{description}
  \item[Mapa de la habitación]
    Es una representación gráfica de la habitación, dibujando sus elementos
    topológicos y su mobiliario.

  \item[Menú de muebles disponibles]
    Menú lateral que contendrá los muebles que se puedan añadir a la
    habitación. Más detalles en \nameref{catalogo_muebles}.
  
  \item[Menú de elementos fijos]
    Menú adyacente al catálogo de muebles disponibles. Mostrará
    \nameref{fijos}, que se puedan añadir a la
    habitación.
\clearpage
  \item[Barra de acciones]
    Es la típica barra situada en la parte superior de la aplicación.
    Permitirá al usuario realizar las siguientes acciones:
    \begin{itemize}
        \item Crear una nueva habitación
        \item Abrir una habitación
        \item Guardar la habitación
        \item Modificar catálogo de muebles
        \item Modificar restricciones
        \item Acceder a la información Acerca de... de la aplicación.
    \end{itemize}
\end{description}

El usuario podrá utilizar el catálogo de muebles y el menú de elementos fijos
para añadir los objetos que desee a su habitación. Para ello solo deberá
seleccionar el mueble o accidente que desee añadir y hacer click en la
posición del mapa donde quiere que se sitúe. La posición será considerada como
correcta si al añadir el mueble en dicha posición se siguen cumpliendo todas
las restricciones definidas en el momento de la inserción. En caso de que la
posición sea incorrecta se le notificará al usuario en forma de mensaje la
restricción que se ha visto comprometida.
\newpar
Sin embargo, el principal objetivo de la aplicación es la generación de
diseños de manera automática. Por lo tanto, el usuario podrá especificar en el
menú de muebles cuáles deberán incluirse en los diseños que genere la
aplicación. Adicionalmente, el usuario podrá fijar la posición de los muebles
en el mapa. Aquellos muebles que estén fijados no serán redistribuidos por la
aplicación al generar los diseños. Por otra parte, los diseños generados deben
cumplir ciertas restricciones  que el usuario podrá añadir y/o modificar a
través del respectivo menú de la barra de acciones.
\newpar
Una vez el usuario haya especificado muebles, elementos fijos y restricciones
podrá pedir que la aplicación genere o termine su diseño cumpliendo con todas
las restricciones (más información en \nameref{restricciones}) que ha
establecido anteriormente. Si no existe ninguna solución que satisfaga las
restricciones se avisará al usuario con un mensaje informativo. Más
información en \nameref{generacion}.

\subsubsection{La habitación}
Una habitación se caracteriza por tener:

\begin{description}
  \item[Dimensiones] Ancho y largo, en centímetros.
  \item[Tipo] Uno de los tipos de habitación presentes en el
  catálogo de habitaciones. Más información en
  \nameref{catalogo_habitaciones}.
\end{description}

\clearpage
El usuario podrá añadir a la habitación:

\begin{description}
  \item[Muebles] Utilizando el menú de muebles disponibles.
  \item[Accidentes] A través del menú de accidentes.
  \item[Restricciones] Ya sean de la habitación en general o de muebles
  específicos. Más información en \nameref{restricciones}.
\end{description}

\subsubsection{El catálogo de tipos habitaciones}
\label{catalogo_habitaciones}
El catálogo de tipos de habitaciones contiene los diferentes tipos de
habitaciones que el usuario puede seleccionar para diseñar su habitación.
\newpar
Se incluye un catálogo de habitaciones por defecto (consultar
\nameref{anexo_habitaciones}) que siempre estará disponible. Si el usuario
realiza cambios sobre el catálogo por defecto y decide guardar los cambios se
le preguntará dónde desea guardar el catálogo. Por ende, el usuario también
podrá cargar catálogos desde un archivo que haya guardado previamente, o bien,
volver al catálogo por defecto.
\newpar
Cada tipo de habitación tiene las siguientes características:

\begin{description}
  \item[Nombre] Por ejemplo: cocina, dormitorio, baño, etc.
  \item[Dimensiones máximas y mínimas] Ancho y largo máximos y mínimos
  que una habitación de este tipo puede tener. En centímetros.
  \item[Restricciones] Puede contener cualquier tipo de restricción aplicable
  a la habitación o a modelos de muebles y sus tipos. Más información en
  \nameref{restricciones}.
\end{description}

\subsubsection{Los elementos fijos}
\label{fijos}
Los elementos fijos son objetos que se pueden situar en la habitación y que
nunca serán distribuidos por la aplicación en la generación de diseños.
Son elementos fijos los accidentes, los radiadores, las ventanas, las
puertas y las chimeneas.
\newpar
Los accidentes representan pilares o pilastros. Un pilastro es un pilar con
almenos una de las caras pegadas a la pared. Se utilizan para definir de
manera más exacta la forma de la habitación que se quiere diseñar.
\newpar
Los elementos fijos tienen unas dimensiones máximas y mínimas y restricciones
propias. Al añadir elementos fijos el usuario debe determinar la posición del
elemento y sus dimensiones.

\subsubsection{Los modelos de muebles}
\label{modelos}
Los modelos de muebles son abstracciones de los muebles que pueden añadirse
a la habitación. Los modelos de muebles están caracterizados por:

\begin{description}
  \item[Nombre] Identifica el modelo.
  \item[Dimensiones] Las dimensiones del mueble: ancho, largo y altura.
  La altura, aunque la habitación se represente en dos dimensiones, puede ser
  utilizada por algunas restricciones.
  \item[Espacio activo] Es el espacio que ocupa el mueble debido a sus
  dimensiones. Este espacio no puede superponerse con el de otros muebles.
  \item[Color]
  \item[Material] El material principal del que está hecho el mueble.
  \item[Precio] Precio del mueble.
  \item[Tipo] Más información en \nameref{tipos}.
\end{description}

\subsubsection{Los tipos de muebles}
\label{tipos}
Los tipos muebles se utilizan para clasificar y definir los modelos más
fácilmente. El tipo de un mueble especifica:
\begin{description}
  \item[Dimensiones máximas y mínimas] En centímetros.
  \item[Utilidades] Define de qué funciones de propósito general dispondrá el
  mueble. El modelo se especializará en función de las utilidades del tipo de
  mueble, adquiriendo nuevas propiedades.
  \item[Restricciones] Inherentes al propio tipo.
\end{description}

El usuario podrá modificar los tipos de muebles disponibles, así como
definir nuevos tipos indicando sus propiedades.
\newpar
Las utilidades disponibles son:
\begin{description}
  \item[Contenedor] El mueble puede ser utilizado para almacenar cosas de
  carácter general (por ejemplo, la papelera o la nevera no sirven para
  guardar cualquier cosa). Añade la propiedad: capacidad, que se mide en
  decimetros cúbicos o litros.
  \item[Asiento] El mueble se puede utilizar para sentarse. Añade la
  propiedad: número de plazas.
  \item[Lugar para dormir] El mueble está diseñado para que las personas
  duerman en él. Añade la propiedad: número de plazas.
  \item[Soporte] Se pueden poner otros muebles encima.
  \item[Trabajo/Comida] El mueble puede utilizarse para trabajar o para comer.
  Se caracteriza porque el mueble se utiliza estando sentado.
\end{description}

Para una lista detallada de los tipos de muebles disponibles y sus propiedades
consultar \nameref{anexo_tipos}.

\subsubsection{El cátalogo de modelos de muebles}
\label{catalogo_muebles}
El catálogo de modelos de muebles representa el conjunto total de modelos de
muebles que están disponibles en la aplicación. El usuario puede añadir y
modificar modelos de muebles del catálogo.
\newpar
Se incluye un catálogo de modelos de muebles por defecto (consultar
\nameref{anexo_muebles}) que siempre estará disponible. Si el usuario realiza
cambios sobre el catálogo por defecto y decide guardar los cambios se le
preguntará dónde desea guardar el catálogo. Por ende, el usuario también podrá
cargar catálogos desde un archivo que haya guardado previamente, o bien,
volver al catálogo por defecto.
\newpar
El menú de modelos de muebles disponibles es un menú lateral que contendrá
todos los muebles que puedan añadirse a la habitación en función de las
restricciones del tipo de habitación que se esté diseñando. En primera
instancia aparecerán los muebles que obligatoriamente deban incluirse en el
diseño (debe haber una cama en un dormitorio). Aparecerán también los muebles
que puedan incluirse en la habitación. Este menú no contendrá los muebles que
no puedan ser incluidos en la habitación (es ilógico tener un horno en un
baño).

\subsubsection{Las restricciones}
\label{restricciones}
Las restricciones son reglas que todo diseño de la habitación debe cumplir.
\newpar
Existen dos clases de restricciones. Las restricciones que introduzca el
usuario según sus preferencias, y las restricciones implícitas que todo diseño
de interiores debe cumplir. El usuario puede introducir tres tipos de
restricciones:
\begin{description}
  \item[De utilidad] El usuario expone las funcionalidades que
  requiere de la habitación. Por ejemplo, puede necesitar que quepan tres
  personas para dormir, o que haya suficiente espacio para guardar toda su
  ropa.
  \item[De diseño] El usuario indica sus preferencias en cuanto
  a materiales y colores, así como en cuanto a presupuesto.
  \item[Espaciales] El usuario da indicaciones sobre la colocación de los
  muebles. Puede imponer una relación entre un par de muebles, así como fijar
  un cierto elemento a una posición.
\end{description}

Las restricciones implícitas funcionan a 3 niveles:
\begin{description}
  \item[A nivel global] Deben cumplirse siempre,   independientemente del
  catálogo de tipos de muebles o del catálogo de   tipos de habitaciones
  usado.
  \item[A nivel de tipos de habitaciones] Se aplican cuando se diseña una
  habitación del tipo en cuestión. Se pueden modificar al crear tipos de
  habitaciones personalizados.
  \item[A nivel de tipos de muebles] Se aplican cuando se usa un mueble del
  tipo en cuestión. Se pueden modificar al crear nuevos tipos de muebles.
\end{description}

Para ver una lista completa de los tipos de restricciones disponibles
consultar \nameref{anexo_restricciones}.

\subsubsection{La generación de diseños}
\label{generacion}
La generación de diseños es la funcionalidad más importante de la aplicación.
El usuario podrá requerir que se ejecute la generación de diseños en cualquier
momento para que la aplicación ofrezca un diseño donde se satisfagan todas las
restricciones que se hayan definido.
\newpar
El diseño generado se aplicará automáticamente al diseño actual del usuario
y el mapa mostrará el nuevo diseño.

\subsubsection{Línea de comandos}
La aplicación podrá utilizarse en modo línea de comandos. En este modo el
usuario utilizará comandos para comunicarse con la aplicación y diseñar su
habitación.
\newpar
Esta funcionalidad será útil para testear toda la aplicación sin necesidad de
una interfaz gráfica, redireccionando la entrada de la aplicación a un fichero
que contenga los comandos y observando la salida generada.

\subsubsection{Extras}
Adicionalmente, si se dispone del tiempo suficiente se implementarán las
siguientes funcionalidades:

\begin{description}
  \item[Enchufes e interruptores] El usuario podrá posicionar, además de
  muebles y accidentes, enchufes e interruptores en la habitación. Los muebles
  tendrán en cuenta estos elementos para no tapar los interruptores y
  posicionarse relativamente cerca de los enchufes en caso de que necesiten
  electricidad para funcionar.

  \item[Deshacer/Rehacer] El usuario podrá deshacer/rehacer los cambios que
  haya efectuado desde que ha empezado a diseñar la habitación, paso por paso.

  \item[Generación de diseños alternativos] El usuario podrá obtener
  diferentes distribuciones de su habitación ejecutando la generación de
  diseños consecutivamente.

  \item[Iluminación] Se añadirá el tipo de mueble iluminar, y se permitirá al
  usuario definir restricciones respecto al nivel de iluminación que quiere en
  su habitación.

  \item[Medidas de calidad] Además de las restricciones la generación de
  diseños tendrá en cuenta la calidad de la distribución generada.
\end{description}
