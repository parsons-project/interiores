Se desea realitzar una aplicación útil para el diseño automático de
habitaciones de diferentes tipos. Para simplificar, podéis suponer que
solo queremos hacer un diseño en 2D (la vista desde arriba).  Cada
tipo de habitación requiere, por sentido común, cierto tipo de
mobiliario. Por ejemplo, una nevera se puede poner en la cocina pero
una bañera no. Un televisor puede ponerse en cualquier tipo de
habitación, incluso el baño. Se puede disponer de mobiliario con
diferentes medidas, colores, texturas, etc. (podéis suponer que los
elementos tienen forma rectangular).  Además, el mobiliario puede
tener restricciones respecto la topologia de la habitación (por
ejemplo, no se puede poner nada delante de una porta, una estanteria
no se puede poner delante de una ventana) o respecto a otro mobiliario
(por ejemplo, un horno no puede estar al lado de una nevera, un
televisor tiene una distancia mínima recomendada respecto el sofá de Y
metros).
\newpar
El usuario indicará el mobiliario que se tiene que añadir.
Adicionalmente, se deberá añadir al diseño el mobiliario que
el usuario no ha requerido pero que, por sentido comun, siempre existe
en una habitación del tipo que se quiere diseñar. Por ejemplo,
supongamos que el usuario quiere que se diseñe su cocina. Quizás no
requiere nada sobre la nevera. Aún así, es de sentido común, hoy en
día, que cualquier cocina disponga de nevera. Por contra, un televisor
no es indispensable en ninguna habitación de una casa.
\newpar
La aplicación debe permitir añadir en 2D la topologia de la habitación
que se quiere diseñar: sus dimensiones (también la podéis suponer
rectangular), la ubicación y dimensiones de ventanas y puertas. También
debe permitir que el usuario defina que elementos quiere que aparezcan en
su habitación. El sistema debe dar como resultado un posible diseño
2D que cumpla las restricciones definidas por el usuario.
