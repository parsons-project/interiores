\subsubsection{Objetivo}
`Diseño de interiores` es un software destinado al diseño de habitaciones
y distribución de su mobiliario. La aplicación es capaz de generar diseños
de habitaciones cumpliendo restricciones definidas por el usuario.

\subsubsection{Flujo de la aplicación}
En este apartado se explica cuál es el flujo de la aplicación, es decir, cuales
son las posibilidades del usuario desde que abre el programa hasta que resuelve
su problema. En los siguientes apartados se explica con mucho más detalle
cada uno de los elementos más importantes que conforman el software.
\newpar
Al iniciar la aplicación se le pregunta al usuario si desea:
\begin{description}
  
  \item[Crear un nuevo diseño]
    En este caso el usuario deberá especificar el tipo de habitación que quiere crear
    y con qué dimensiones.
  
  \item[Abrir un diseño ya existente]
    El usuario podrá abrir diseños que haya guardado préviamente. En este caso se cargará
    la habitación en el mismo estado que cuando se guardó para que el usuario pueda seguir
    trabajando en su diseño.

\end{description}

Una vez el usuario haya creado o abierto una habitación, la aplicación pasará a mostrarle
diversos elementos para facilitarle el diseño de la habitación. Estos son:

\begin{description}
  \item[Mapa de la habitación]
    Es una representación gráfica de la habitación, dibujando sus accidentes y muebles.
    Su tamaño varía en función de las dimensiones de la habitación.
    Se encontrará en el centro de la aplicación y tendrá un fondo blanco.

  \item[Cátalogo de muebles]
    Menú lateral que contendrá todos los muebles que pueden añadirse a la habitación en función
    del tipo de habitación que se esté diseñando.
    En primera instancia aparecerán los muebles que obligatoriamente deben incluirse en el diseño (debe haber
    una cama en un dormitorio).
    A continuación, estarán aquellos muebles que no son de inclusión obligatoria, pero son comunes
    en el tipo de habitación (un lavavajillas en la cocina, por ejemplo).
    Finalmente aparecerán todos los demás muebles que pueden incluirse en la habitación.
    Este menú no contendrá los muebles que NO pueden ser incluidos en la habitación (es ilógico tener un
    horno en un baño).

\end{description}

\subsubsection{La habitación}

\subsubsection{Los accidentes}

\subsubsection{El cátalogo de muebles}

\subsubsection{Las restricciones}

\subsubsection{La generación de diseños}

