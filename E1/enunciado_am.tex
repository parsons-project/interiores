`Diseño de interiores' es un software que permite al usuario diseñar
una habitación y distribuir su mobiliario. El programa dispone de
varias herramientas, tanto visuales como algorítmicas, que ayudan al
usuario a obtener una distribución óptima del mobiliario bajo unas
restricciones altamente personalizables.
\newpar
Antes de comenzar, haremos una distinción entre:
\begin{description}
\item[Elementos fijos] Esto es, las características que se determinan
  en el momento de construir la habitación, y no pueden modificarse
  sin pasar por una reforma estructural.  Dentro de esta categoría
  entran: las dimensiones de la habitación, las columnas y otros tipos
  de accidentes que hacen variar su forma rectangular, las puertas,
  las ventanas, y los radiadores.  Adicionalmente podrían tenerse en
  cuenta interruptores y enchufes como propiedades fijas.
\item[Elementos desplazables] Es decir, los elementos que pueden
  moverse libremente por el espacio libre de la habitación. Dentro de
  esta categoría se encuentran los muebles y los electrodomésticos.
\end{description}
\newpar Atendiendo a estas dos posibilidades, podemos encontrarnos con
dos grupos bien diferenciados de usuarios:
\begin{itemize}
\item Aquellos que estén diseñando una vivienda desde cero, y por lo
  tanto tengan control sobre los elementos fijos de la habitación.
\item Aquellos que estén delante de una habitación ya construida, y
  por lo tanto necesiten respetar los elementos fijos de la misma,
  pudiendo simplemente variar la distribución de los elementos
  desplazables.
\end{itemize}
Con el objetivo de crear una solución general que sea de ayuda para
ambos perfiles de usuario, ofreceremos al usuario herramientas para
modificar tanto los elementos fijos como los elementos desplazables de
la habitación.
\begin{center}
\rule{0.75\linewidth}{0.5mm} % Linea que tienes puesta
\end{center}
En primer lugar, el usuario deberá escoger la clase de habitación que
desea diseñar, de entre uno de los siguientes tipos: Dormitorio,
Estudio, Cocina, Comedor, Salón y Baño.
\newpar
En segundo lugar, se preguntará al usuario cuáles son las dimensiones
de la habitación que va a diseñar.  Existen unas dimensiones mínimas
permitidas, las cuales dependen de los muebles que por sentido común
consideramos esenciales en una habitación y, por ende, dependen del
tipo de esta. Por ejemplo, en un dormitorio es esencial tener una
cama. En un baño es esencial tener una ducha/bañera y un retrete. En
una cocina es esencial tener una pila, etc. Asimismo, ninguna
habitación podrá superar los 50m$^2$ de superficie. El diseño de
interiores se hará sobre dos dimensiones (no consideraremos la
altura).
\newpar
A continuación, se mostrará al usuario la planta de la habitación (
que formará un rectángulo), y se le pedirá que coloque los elementos
fijos de esta: Puertas, Ventanas, Radiadores, y Accidentes
estructurales. Entendemos por accidentes estructurales todo tipo de
elementos que modifiquen la forma rectangular de la habitación. Estos
elementos también serán, a su vez, de forma rectangular.
\newpar
Una vez colocados los elementos fijos de la habitación, se le
presentará al usuario:
\begin{itemize}
\item Una lista de muebles que podría contener la habitación.
\item La vista de planta del diseño actual.
\end{itemize}
Para entender el modo en que el programa ayuda al usuario es vital
estudiar el funcionamiento de estos dos elementos.

\subsubsection*{Vista de planta} % el asterisco es para que no lo numere
Inicialmente, la vista de planta muestra la habitación que el usuario
ha configurado hasta el momento, incluyendo los elementos fijos. Las
dimensiones de la habitación se escalarán de modo que el usuario pueda
visualizar la superficie completa en todo momento. 
\newpar
Ahora, cada vez que se coloca un mueble en la vista de la habitación,
el usuario puede arrastrarlo libremente por su superficie. El programa
le marcará en verde la superficie sobre la que puede soltarlo, y en
rojo aquella sobre la que no puede, debido a restricciones de
cualquier tipo. Además, el usuario puede marcar un mueble (que esté en
una posición válida) como fijo. Esto hará que el algoritmo de
redistribución tenga en cuenta el deseo del usuario de dejar el mueble
en tal lugar, pase lo que pase.
\subsubsection*{Lista de muebles} 
La lista contiene los muebles que el usuario podría poner en su
habitación, ordenados por prioridad.  Inicialmente, la lista contiene
muebles sugeridos por el software. La tipología de los muebles depende
del tipo de habitación que el usuario esté diseñando, mientras que la
cantidad de muebles sugeridos depende del espacio útil que haya
quedado en la habitación descontando de sus dimensiones el espacio
ocupado por los elementos fijos.
\newpage
\noindent Respecto de la lista, el usuario puede:
\begin{enumerate}[label=\textbf{\alph*)}] % letras como contador a), b)
\item Añadir o Quitar muebles de la lista según sus preferencias.
\item Reordenarlos según sus prioridades.
\end{enumerate}
Ahora, respecto de los muebles, el usuario puede:
\begin{enumerate}[label=\textbf{\alph*)}]
\item Modificar un mueble en concreto, cambiando su tamaño o su
  aspecto (color, dibujo). Su tamaño, igual que el de las
  habitaciones, estará acotado entre un mínimo y un máximo.
\item Marcarlos como activos o inactivos.
\end{enumerate}
Cada vez que el usuario marca un mueble como activo, el programa
intenta buscarle un hueco en la habitación. Para ello, redistribuye
todos los elementos desplazables de la misma, incluido el recién
activado. Cada vez que el usuario marca un mueble como inactivo, el
programa lo elimina del diseño de la habitación y redistribuye los
demás muebles acorde al espacio que el reci én desactivado deja libre.
\newpar 
N.B. Si el programa fuese lo suficientemente rápido, esto se
produciría de forma instantánea cada vez que un usuario
activase/desactivase un mueble. Si no fuera posible hacerlo de manera
instantánea, habría un botón para aplicar esta función, que el usuario
pulsaría después de haber hecho los cambios oportunos en la lista.
\newpar
Es importante entender que añadir/quitar elementos de la lista, y
marcarlos como activo/inactivo son dos funciones totalmente
distintas. La primera determina si un elemento es del interés del
usuario en aquella habitación. La segunda permite jugar con la
distribución de un subconjunto de los muebles que le interesan al
usuario.
\newpar
Una vez el usuario tiene el diseño final que quiere para su
habitación, dispone de las siguientes opciones:

\begin{enumerate}
\item Exportarlo a formato PDF.
\item Guardar el diseño como un archivo propio del programa, de modo
  que pueda volver a cargarlo y seguir haciendo modificaciones en otro
  momento.
\item Salir del programa. Si no ha guardado los cambios, al salir se
  le preguntará si quiere hacerlo.
\end{enumerate}
\textbf{Posibles ampliaciones del programa}
\newpar
Una posibilidad de ampliación del software presentado sería la de
incluir la posibilidad de importar tipos de habitaciones y tipos de
muebles personalizados a partir de archivos de texto con un
determinado formato. Esto permitiría a terceras personas desarrollar
plug-ins que amplíen el campo de aplicación del programa,
aprovechando su motor de redistribución y diseño.
\newpar
Un ejemplo de esto sería el de un módulo que permitiese diseñar
jardines, y añadiera los elementos propios de un jardín.
\newpar
\noindent \textbf{\emph{\LARGE Faltan los tipos de restricciones.}}
