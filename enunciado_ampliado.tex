\documentclass[a4paper,11pt]{article}
\usepackage[utf8]{inputenc}


\title{Diseño Interiores}
\author{}

\begin{document}
\maketitle
\section{Enunciado Original}
Es  desitja realitzar una aplicació útil per al disseny automàtic d'habitacions de diferents tipus. 
Per simplificar, podeu suposar que només volem fer un disseny en 2D (la vista des de dalt). 
Cada tipus d'habitació requereix, per sentit comú,  cert tipus de mobiliari. Per exemple, una 
nevera es pot posar a la cuina però una banyera no. Un televisor pot posar-se en qualsevol 
tipus d'habitació, inclòs el bany. Es pot disposar  de mobiliari amb diferents mides, colors, 
textures, etc. (podeu suposar que els elements tenen forma rectangular). A més a més, el 
mobiliari pot tenir restriccions respecte la topologia de l'habitació (per exemple, no es pot posar 
res davant d’una porta, una prestatgeria no es pot posar davant d’una finestra) o respecte a 
altre mobiliari (per exemple, un forn no pot estar al costat d'una nevera, un televisor té una 
distància mínima recomanada respecte el sofà de Y metres).

L’usuari indicarà el mobiliari que s'ha d'afegir. Addicionalment, s'haurà d'afegir al disseny aquell 
mobiliari que l’usuari no ha requerit però que, per sentit comú, sempre existeix en una habitació 
del tipus que es vol dissenyar. Per exemple, suposem que l’usuari vol que es dissenyi la seva 
cuina. Potser no requereix res sobre la nevera. Tot i així, es de sentit comú, avui dia, que 
qualsevol cuina disposi de nevera. Per contra, un televisor no es indispensable en cap habitació 
d'una casa.

L'aplicació ha de permetre introduir en 2D la topologia de l'habitació que es vol dissenyar: les 
seves dimensions (també la podeu suposar rectangular), la ubicació i dimensions de finestres i 
portes. També ha de permetre que l’usuari defineixi quins elements vol que apareguin a la seva 
habitació. El sistema ha de donar com a resultat un possible disseny 2D que compleixi les 
restriccions definides per l’usuari.

\section{Enunciado Ampliado}


\end{document}