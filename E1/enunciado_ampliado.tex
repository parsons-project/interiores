\subsubsection{Objetivo}
`Diseño de interiores' es un software destinado al diseño de habitaciones
y distribución de su mobiliario. La aplicación es capaz de generar diseños
de habitaciones cumpliendo con una serie de restricciones, permitiendo
al usuario definir cuáles y cómo deben ser estas restricciones.

\subsubsection{Flujo de la aplicación}
En este apartado se explica cuál es el flujo de la aplicación, es decir,
cuáles son las posibilidades del usuario desde que abre el programa hasta que
resuelve su problema. En los siguientes apartados se explica con mucho más
detalle cada uno de los elementos más importantes que conforman el software.

\newpar
Al iniciar la aplicación se le pregunta al usuario si desea:
\begin{description}
  \item[Crear un nuevo diseño]
    En este caso el usuario deberá especificar el tipo de habitación que
    quiere crear y con qué dimensiones.

  \item[Abrir un diseño ya existente]
    El usuario podrá abrir diseños que haya guardado previamente. En este caso
    se cargará la habitación en el mismo estado que cuando se guardó para que
    el usuario pueda seguir trabajando en su diseño.
\end{description}

Una vez el usuario haya creado o abierto una habitación, la aplicación pasará
a mostrarle diversos elementos para facilitarle el diseño de la habitación.
Estos son:

\begin{description}
  \item[Mapa de la habitación]
    Es una representación gráfica de la habitación, dibujando sus accidentes y
    muebles. Su tamaño varía en función de las dimensiones de la habitación.
    Se encontrará en el centro de la aplicación y tendrá un fondo blanco.

  \item[Menú de muebles disponibles]
    Menú lateral que contendrá los muebles que se puedan añadir a la
    habitación. Más detalles en \nameref{catalogo_muebles}.
  
  \item[Menú de elementos fijos]
    Menú adyacente al catálogo de muebles disponibles. Mostrará
    \nameref{fijos}, que se puedan añadir a la
    habitación.
\clearpage
  \item[Barra de acciones]
    Es la típica barra situada en la parte superior de la aplicación.
    Permitirá al usuario realizar las siguientes acciones:
    \begin{itemize}
        \item Crear una nueva habitación
        \item Abrir una habitación
        \item Guardar la habitación
        \item Modificar catálogo de muebles
        \item Modificar restricciones
        \item Acceder a la información Acerca de... de la aplicación.
    \end{itemize}
\end{description}

El usuario podrá utilizar el catálogo de muebles y el menú de elementos fijos
para añadir los objetos que desee a su habitación. Para ello solo deberá
seleccionar el mueble o accidente que desee añadir y hacer click en la
posición del mapa donde quiere que se sitúe. La posición será considerada como
correcta si al añadir el mueble en dicha posición se siguen cumpliendo todas
las restricciones definidas en el momento de la inserción. En caso de que la
posición sea incorrecta se le notificará al usuario en forma de mensaje la
restricción que se ha visto comprometida.
\newpar
Sin embargo, el principal objetivo de la aplicación es la generación de
diseños de manera automática. Por lo tanto, el usuario podrá especificar en el
menú de muebles cuáles deberán incluirse en los diseños que genere la
aplicación. Adicionalmente, el usuario podrá fijar la posición de los muebles
en el mapa. Aquellos muebles que estén fijados no serán redistribuidos por la
aplicación al generar los diseños. Por otra parte, los diseños generados deben
cumplir ciertas restricciones  que el usuario podrá añadir y/o modificar a
través del respectivo menú de la barra de acciones.
\newpar
Una vez el usuario haya especificado muebles, accidentes y restricciones podrá
pedir que la aplicación genere o termine su diseño cumpliendo con todas las
restricciones (más información en \nameref{restricciones}) que ha establecido
anteriormente. Si no existe ninguna solución que satisfaga las restricciones
se avisará al usuario con un mensaje informativo. Más información en
\nameref{generacion}.

\subsubsection{La habitación}
Una habitación se caracteriza por tener:

\begin{description}
  \item[Dimensiones] Ancho y largo, en centímetros.
  \item[Tipo] Uno de los tipos de habitación presentes en el
  catálogo de habitaciones. Más información en
  \nameref{catalogo_habitaciones}.
\end{description}

\clearpage
El usuario podrá añadir a la habitación:

\begin{description}
  \item[Muebles] Utilizando el menú de muebles disponibles.
  \item[Accidentes] A través del menú de accidentes.
  \item[Restricciones] Ya sean de la habitación en general o entre muebles
  específicos. Más información en \nameref{restricciones}.
\end{description}

\subsubsection{El catálogo de habitaciones}
\label{catalogo_habitaciones}
El catálogo de habitaciones contiene los diferentes tipos de habitaciones que
el usuario puede seleccionar para diseñar su habitación.
\newpar
Se incluye un catálogo de habitaciones por defecto (consultar
\nameref{anexo_habitaciones}) que siempre estará disponible. Si el usuario
realiza cambios sobre el catálogo por defecto y decide guardar los cambios se
le preguntará dónde desea guardar el catálogo. Por ende, el usuario también
podrá cargar catálogos desde un archivo que haya guardado previamente, o bien,
volver al catálogo por defecto.
\newpar
Cada tipo de habitación tiene las siguientes características:

\begin{description}
  \item[Nombre] Por ejemplo: cocina, dormitorio, baño, etc.
  \item[Dimensiones mínimas] Ancho y largo mínimos que una habitación de este
  tipo puede tener. En centímetros.
  \item[Dimensiones máximas] Ancho y largo máximos que una habitación de este
  tipo puede tener. En centímetros.
  \item[Restricciones] Puede contener cualquier tipo de restricción aplicable
  a la habitación o a modelos de muebles y sus tipos. Más información en
  \nameref{restricciones}.
\end{description}

\subsubsection{Los elementos fijos}
\label{fijos}
Los elementos fijos son objetos que se pueden situar en la habitación y que
nunca serán distribuidos por la aplicación en la generación de diseños.
Son elementos fijos los accidentes, las ventanas y las puertas.
\newpar
Los accidentes representan pilares o pilastros. Se utilizan para definir de
manera más exacta la forma de la habitación que se quiere diseñar.
\newpar
Al añadir elementos fijos el usuario debe determinar la posición del elemento
y sus dimensiones. En el mapa de la habitación los accidentes tienen forma
rectangular, mientras que las puertas y ventanas se representan gráficamente
igual que en diseños arquitectónicos.

\subsubsection{Los modelos de muebles}
\label{modelos}
Los modelos de muebles son abstracciones de los muebles que pueden añadirse
a la habitación. Los modelos de muebles están caracterizados por:

\begin{description}
  \item[Nombre] Identifica el modelo.
  \item[Dimensiones] Las dimensiones del mueble: ancho, largo y altura.
  La altura, aunque la habitación se represente en dos dimensiones, puede ser
  utilizada por algunas restricciones.
  \item[Espacio activo] Es el espacio que ocupa el mueble debido a sus
  dimensiones. Este espacio no puede superponerse con el de otros muebles.
  \item[Color]
  \item[Material] El material del que está hecho el mueble.
  \item[Precio] Precio orientativo del mueble.
  \item[Capacidad] Litros que se pueden almacenar en el mueble.
  \item[Plazas] Máximo número de personas que pueden utilizar el mueble al
  mismo tiempo.
  \item[Tipo] Grupo al que pertenece el mueble. Por ejemplo: cama, silla,
  mesa, sofá, etc.
  \item[Utilidad] Uso que se le puede dar al mueble. Un mueble puede tener
  diversas utilidades. Por ejemplo: almacenamiento, trabajo, soporte, etc.
  \item[Restricciones] Inherentes al propio mueble. Más detalles en
  \nameref{restricciones}.
\end{description}

\subsubsection{El cátalogo de muebles}
\label{catalogo_muebles}
El catálogo de muebles representa el conjunto total de modelos de muebles que
están disponibles en la aplicación. El usuario puede añadir y modificar
modelos de muebles del catálogo.
\newpar
Se incluye un catálogo de muebles por defecto (consultar
\nameref{anexo_muebles}) que siempre estará disponible. Si el usuario realiza
cambios sobre el catálogo por defecto y decide guardar los cambios se le
preguntará dónde desea guardar el catálogo. Por ende, el usuario también podrá
cargar catálogos desde un archivo que haya guardado previamente, o bien,
volver al catálogo por defecto.
\newpar
El menú de muebles disponibles es un menú lateral que contendrá todos los
muebles que puedan añadirse a la habitación en función de las restricciones
del  tipo de habitación que se esté diseñando. En primera instancia aparecerán
los muebles que obligatoriamente deban incluirse en el diseño (debe haber una
cama en un dormitorio). A continuación, estarán aquellos muebles que no son de
inclusión obligatoria, pero son comunes en el tipo de habitación (un
lavavajillas en la cocina, por ejemplo). Finalmente aparecerán todos los demás
muebles que puedan incluirse en la habitación. Este menú no contendrá los
muebles que NO puedan ser incluidos en la habitación (es ilógico tener un
horno en un baño).

\subsubsection{Las restricciones}
\label{restricciones}
Las restricciones son reglas que todo diseño de la habitación debe cumplir.
Las restricciones pueden relacionar diversos modelos de muebles o instancias
de muebles entre sí.
\newpar
Sin embargo, la habitación también puede contener restricciones aplicables a:
\begin{description}
  \item[Todos los muebles que vayan a añadirse a la habitación] El usuario
  puede indicar que quiere que todos los muebles sean de un mismo color, por
  ejemplo.
  \item[Tipos de muebles] La habitación puede contener restricciones que
  obliguen o prohiban la inclusión de cierto tipo de mueble al diseño, por
  ejemplo.
  \item[Modelos de muebles] Cuando el usuario indica que un modelo de mueble
  debe incluirse en su diseño, por ejemplo.
\end{description}

Para ver una lista completa de los tipos de restricciones disponibles
consultar \nameref{anexo_restricciones}.

\subsubsection{La generación de diseños}
\label{generacion}
La generación de diseños es la funcionalidad más importante de la aplicación.
El usuario podrá requerir que se ejecute la generación de diseños en cualquier
momento para que la aplicación ofrezca un diseño donde se satisfagan todas las
restricciones que se hayan definido.
\newpar
El diseño generado se aplicará automáticamente al diseño actual del usuario
y el mapa mostrará el nuevo diseño.

\subsubsection{Línea de comandos}
La aplicación podrá utilizarse en modo línea de comandos. En este modo el
usuario utilizará comandos para comunicarse con la aplicación y diseñar su
habitación.
\newpar
Esta funcionalidad será útil para testear toda la aplicación sin necesidad de
una interfaz gráfica, redireccionando la entrada de la aplicación a un fichero
que contenga los comandos y observando la salida generada.
\newpar
La interfaz gráfica se construirá encima de la línea de comandos, de manera
que toda acción que haga el usuario en la interfaz gráfica se traducirá a
comandos que serán interpretados por la aplicación. De este modo, al guardar
un diseño de una habitación se guardarán los comandos empleados hasta ese
punto y al cargarlo se ejecutarán dichos comandos. Más detalles en
\nameref{descripcion_casos}.

\subsubsection{Extras}
Adicionalmente, si se dispone del tiempo suficiente se implementarán las
siguientes funcionalidades:

\begin{description}
  \item[Enchufes e interruptores] El usuario podrá posicionar, además de
  muebles y accidentes, enchufes e interruptores en la habitación. Los muebles
  tendrán en cuenta estos elementos para no tapar los interruptores y
  posicionarse relativamente cerca de los enchufes en caso de que necesiten
  electricidad para funcionar.

  \item[Deshacer/Rehacer] El usuario podrá deshacer/rehacer los cambios que
  haya efectuado desde que ha empezado a diseñar la habitación, paso por paso.

  \item[Generación de diseños alternativos] El usuario podrá obtener
  diferentes distribuciones de su habitación ejecutando la generación de
  diseños consecutivamente.
\end{description}
