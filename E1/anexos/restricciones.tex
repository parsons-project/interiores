\section{Anexo I. Tipos de restricciones disponibles}
\label{anexo_restricciones}

\subsection{Habitación - Elemento fijo}
Son los que relacionan la habitación con algun elemento fijo que
contiene. Ejemplos:
\begin{itemize}
\item Una habitación debe tener almenos una puerta.
\item Un elemento topológico no se puede mover una vez colocado.
\item ...
\end{itemize}
\subsection{Elemento - Elemento}
Relacionan elementos entre ellos. Ejemplos:
\begin{itemize}
\item Un mueble con funcionalidad comida/trabajo debe tener cerca
de un mueble con funcionalidad de asiento.
\item El espacio activo de dos elementos no se puede sobreponer,
excepto por un objeto encima de otro con utilidad soporte
de dimensiones mayores.
\item La tele debe estar enfrente del sofa.
\item La nevera no puede estar cerca del horno.
\item El espacio pasivo de un elemento no se puede sobreponer
con el espacio activo de otro.
\item Dos elementos deben estar a una distancia X el uno del otro.
\item ...
\end{itemize}
\subsection{Habitación - Elemento}
Relacionan el global de la habitación con algun elemento genérico que
contenga. Ejemplos:
\begin{itemize}
\item El espacio pasivo de todo elemento debe ser accesible andando (con un
pasaje libre de almenos 50cm de ancho) desde todas las puertas.
\item Ningún mueble puede salirse del perímetro de la habitación
\item ...
\end{itemize}
\subsection{Mueble - Tipo Mueble}
Relacionan los muebles con su tipo de mueble. Ejemplos:
\begin{itemize}
\item El mueble no puede tener dimensiones fuera del rango permitido.
\item El mueble debe tener la funcionalidad que le indica su tipo.
\item ...
\end{itemize}
\subsection{Tipo Mueble - Tipo habitación}
\begin{itemize}
\item Todo modelo de mueble de la habitación debe tener un tipo
existente en el catálogo de tipos de muebles.
\item Todo modelo de mueble de la habitación debe tener un tipo que pueda
ir en el tipo de habitación de la habitación.
\item La habitación debe contener modelos de muebles de todos los tipos de
muebles que deben ir en el tipo de habitación de la habitación.
\item Un tipo de mueble respecto un tipo de habitación, tiene uno y sólo uno
de los estados siguientes:
\begin{itemize}
 \item Básico
 \item Puede ir
 \item No debe ir
\end{itemize}
\end{itemize}

\subsection{Definidas por el usuario}
Puede definir restricciones de diseño (colores, materiales),
generales para la habitación (tamaños, topología, marcar como fijo un
elemento), de precio (máximo, mínimo) o capacidad (que puedan dormir 2
personas).
