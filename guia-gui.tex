\documentclass[a4paper, 11pt]{article}
\usepackage[utf8]{inputenc}


\begin{document}

En este documento se especifica como usar la interf\'{i}cie gr\'{a}fica del
programa Interiores.

\section{Menu}
\subsection{File}
\subsubsection{New Room Design}
Crea un nuevo dise\~{n}o. Abre una nueva ventana en la que se piden los
par\'{e}metros de la habitaci\'{o}n teniendo en cuenta el cat\'{a}logo actual.
Dado un tipo se muestran los muebles que deben ir por obligaci\'{o}n y los
muebles que no pueden ir en ella.
\subsubsection{Open Room Design}
Abre un dise\~{n}o desde el disco. Se abre una nueva ventana que pide el
\emph{path} del archivo y lo carga al mapa de dise\~{n}os.
\subsubsection{Save Room Design}
Guarda un dise\~{n}o ya creado a un archivo de disco. Este podr\'{a} ser
recuperado luego. Avisa de sobreescritura en caso de que el archivo ya
exista. El dise\~{n}o se guarda completamente: los muebles y su posici\'{o}n,
las restricciones y los elementos fijos.
\subsubsection{Exit}
Sale del programa. Cierra todas las ventanas y acaba con todos los
procesos que estaba usando.

\subsection{Edit}
\subsubsection{Room types catalog}
Abre el editor de cat\'{a}logos de tipos de habitaciones. En \'{e}l se puede
crear nuevos tipos y modificar o eliminar ya existentes. Se piden
par\'{a}metros necesarios para el programa como tama\~{n}o m\'{i}nimo y m\'{a}ximo,
elementos prohibidos y obligatorios... Pueden crearse nuevos cat\'{a}logos
y guardarlos en disco para posteriormente recuperarlos.
\subsubsection{Furniture types catalog}
Abre el editor de tipos de mueble. En \'{e}l se puede crear nuevos tipos
de muebles, modificar o eliminar ya existentes. Se piden valores como
funcionalidad, dimensiones, si debe ir o no pegado a una pared... Los
cat\'{a}logos generados pueden ser guardados en disco y luego ser
cargados.
\subsubsection{Furniture models catalog}
Abre el editor de modelos de un tipo de mueble concreto. Se pueden
introducir los atributos de un mueble, as\'{i} como su tama\~{n}o, precio,
color... Los cat\'{a}logos se pueden modificar y guardar en disco para
luego ser recuperados.

\subsection{Tools}
\subsubsection{Select}
Esta herramienta permite seleccionar elementos ya colocados en el mapa
mediante click al bot\'{o}n izquierdo del rat\'{o}n. Un click derecho sobre el
mueble colocado abre su editor de restricciones.
\subsubsection{Move}
Esta herramienta permite mover libremente los elementos seleccionados
por la habitaci\'{o}n.  Para ello se debe clickar el bot\'{o}n izquierdo del
rat\'{o}n y desplazarlo mientras se mantiene pulsado.
\subsubsection{Door}
Esta herramienta permite a\~{n}adir puertas a la habitaci\'{o}n. Para ello
debe clickarse con bot\'{o}n izquierdo y la puerta ser\'{a} a\~{n}adida a la pared
m\'{a}s cercana. Si se mantiene pulsado y se arrastra en cualquier
direcci\'{o}n se podr\'{a} definir el tama\~{n}o de la puerta as\'{i} como su
sentido. Se puede cancelar su creaci\'{o}n, si mientras se est\'{e} pulsando
el bot\'{o}n izquierdo del rat\'{o}n se pulsa la tecla 'ESC'.  En el momento
de dejar ir el bot\'{o}n del rat\'{o}, la puerta ser\'{a} a\~{n}adida a la habitaci\'{o}n
autom\'{a}ticamente.
\subsubsection{Window}
Esta herramienta permite a\~{n}adir ventanas a la habitaci\'{o}n. Para ello
debe clickarse con bot\'{o}n izquierdo y la ventana ser\'{a} a\~{n}adida a la
pared m\'{a}s cercana. Si se mantiene pulsado y se arrastra en cualquier
direcci\'{o}n se podr\'{a} definir el tama\~{n}o de la ventana. Se puede cancelar
la operaci\'{o}n pulsando 'ESC' mientras se est\'{a} arrastrando el rat\'{o}n.
\subsubsection{Pillar}
Esta herramienta permite a\~{n}adir bloques macizos para representar
columnas o para simular habitaciones no rectangulares.  Para colocar
un bloque basta con hacer click con el bot\'{o}n izquierdo del rat\'{o}n y
arrastrar en cualquier direcci\'{o}n.  Si se pulsa la tecla 'ESC' durante
el proceso, \'{e}ste ser\'{a} cancelado. En el momento de dejar ir el bot\'{o}n,
si no se ha cancelado la operaci\'{o}n un nuevo bloque ser\'{a} a\~{n}adido a la
habitaci\'{o}n.

\subsection{View}
\subsubsection{Terminal}
Abre una terminal virtual conectada con el programa. En ella se pueden
escribir los comandos definidos en \emph{Help}: \emph{Terminal} (pulsando F2).

\section{Map}
\subsection{Tools bar}
Es el panel lateral izquierdo. Permite seleccionar r\'{a}pidamente las
herramientas anteriormente descritas, haciendo click en los botones
con el icono de cada herramienta.
\subsection{Grid}
Es el panel central de la ventana. En el se muestran los dise\~{n}os, as\'{i}
como la informaci\'{o}n del \emph{debugger}.  Las operaciones que permiten
las herramientas se ejecutan sobre est\'{a} parte de la ventana. Los
muebles colocados pueden ser movidos, seleccionados... Click con el
bot\'{o}n derecho sobre un mueble colocado, mostrar\'{a} el editor de
restricciones de ese mueble. Se pueden suprimir los muebles
seleccionados pulsando la tecla 'SUPR'.
\subsection{Wishlist}
En la parte derecha superior de la ventana se muestran los muebles
disponibles para a\~{n}adir a la habitaci\'{o}n.  Efectivamente, no se
muestran los muebles prohibidos para ese tipo de habitaci\'{o}n. En la
parte inferior se muestran los ya elejidos que contienen desde el
principio los tipos obligatorios para esa habitaci\'{o}n. Si se hace doble
click (o bien intro sobre el seleccionado) se accede al editor de
restricciones de ese mueble en concreto.  Para eliminar un mueble de
entre los elegidos, se debe seleccionar y seguidamente pulsar
'SUPR'. Si el elemento ya se estaba mostrando en el mapa, \'{e}ste ser\'{a}
actualizado consecuentemente.
\subsection{Solve}
Los botones de la parte inferior derecha sirven para lanzar el
algoritmo con posibles opciones. Si la \emph{checkbox} de {\tt Time}
est\'{a} activa, se cronometrar\'{a} el tiempo tardado para generar la
habitaci\'{o}n. Si la opci\'{o}n {\tt Debug} est\'{a} activada se abrir\'{a} el
debugger. Una vez se ha encontrado soluci\'{o}n, si se vuelve a accionar
el bot\'{o}n de 'Solve Design', el algoritmo continuar\'{a} desde donde lo dejo.
\section{Constraints editor}
Esta ventana permite editar las restricciones de un mueble concreto.
Se puede acceder a \'{e}l mediante doble click sobre el elemento elegido,
con la tecla 'ENTER' o bien con el bot\'{o}n derecho sobre el elemento del
mapa al que se le quieran editar las restricciones. En este editor se
pueden a\~{n}adir y/o quitar todas las restricciones disponibles excepto
las globales.  Una vez elegidos los par\'{a}metros, al pulsarse 'Add
Constraint', se a\~{n}adir\'{a} la restricci\'{o}n a ese elemento del dise\~{n}o en
concreto. La parte superior derecha muestra las restricciones unarias,
la central las binarias, y la inferior las que vienen por defecto en
el mueble por ser del tipo que es. En esta ventana tambi\'{e}n se indica
si es un mueble que va pegado a una pared por defecto. Una vez
a\~{n}adidas todas la restricciones que se quieran se puede pulsar 'Close'
para cerrar la ventana.  Para quitar restricciones se puede
seleccionar la restricci\'{o}n correpondiente y pulsar la tecla 'SUPR'.
\section{Catalog Editor}
Los tres tipos de cat\'{a}logos (tipo de habitaci\'{o}n, tipo de mueble y modelo de mueble) comparten
una interfaz similar. Los elementos comunes m\'{a}s destacables son:
\begin{description}
\item[La persistencia] Todos los cat\'{a}logos pueden guardarse y
  recuperarse de disco. Para ello pueden usarse los botones de 'Save
  Catalog' y 'Load Catalog'. Adem\'{a}s se pueden crear nuevos cat\'{a}logos
  pulsando el bot\'{o}n de 'New Catalog' o bien eliminarlos pulsando el
  bot\'{o}n 'Remove Catalog'.
\item[A\~{n}adir] Para a\~{n}adir nuevos elementos a un cat\'{a}logo basta con
  hacer click en el bot\'{o}n con el s\'{i}mbolo '+' de la parte inferior.
\item[Eliminar] Para eliminar un elemento del cat\'{a}logo basta con hacer
  click en el bot\'{o}n con una cruz del lado izquierdo del mueble que se
  desee eliminar.
\item[Modificar] Para modificar un elemento del cat\'{a}logo basta con
  escribir sobreescribir el campo que se quiera con el valor deseado.
\end{description}

\section{Debugger}
Esta ventana permite ejecutar el algoritmo con control absoluto. Se
muestra el valor que tiene cada variable en cada momento. Se muestran
las posiciones que prueba cada mueble (con colores distintos para cada
uno). Se puede ejecutar el algoritmo a tramos con 'Resume'/'Pause' y
parar la ejecuci\'{o}n con 'Stop' o bien iterar paso a paso indicando el
numero de iteraciones que debe hacer y pulsando 'Iterate'.

\end{document}
