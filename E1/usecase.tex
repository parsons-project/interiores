\subsubsection{\large{Diseñar una habitación}}
Engloba las acciones relacionadas con el diseño de
la habitación. El actor es siempre el usuario.
\begin{description}
\item[Cargar diseño] Permite recuperar el estado
  de un diseño anteriormente guardado.
\item[Especificar la topología] Determinar la topología de
  la habitación a diseñar.
  \begin{description}
    \item[Especificar el tipo] El usuario escoge un tipo de habitación de
      entre todos los disponibles.
    \item[Especificar las dimensiones] El usuario indica las medidas de la
      habitación. Ésta tendrá forma rectangular.
    \item[Colocar elementos fijos] El usuario selecciona un elemento
      fijo de entre los disponibles y lo sitúa en la habitación.
  \end{description}
\item[Escoger muebles] De entre una lista de muebles que pueden
  encontrarse dentro de la habitación, el usuario escoge aquellos que
  quiere que se tengan en cuenta cuando se genere el diseño de la
  misma. El programa podrá añadir muebles a un diseño, aunque el
  usuario no los haya escogido, si deben aparecer de forma obligatoria
  en el tipo de habitación seleccionado.
\item[Especificar restricciones]
  Gestionar las restricciones de la habitación así como
  de los elementos que contiene.
  \begin{description}
  \item[Añadir restricción] El usuario añade una restricción a tener
    en cuenta a la hora de diseñar la habitación. Esta restricción
    será escogida de entre unos tipos determinados ofrecidos por el
    programa, y tendrá asociado un cierto número de parámetros.
  \item[Modificar restricción] El usuario selecciona una
    restricción existente y modifica los parámetros con los
    que se aplica dicha restricción.
  \item[Quitar restricción] El usuario selecciona una restricción
    existente y la elimina del sistema.
  \end{description}
\item[Generar una distribución automática de los elementos] El usuario
  indica al programa que le muestre un diseño posible de la
  habitación. Este tendrá en cuenta la lista de muebles que el usuario
  desea poner en la habitación, y las restricciones que quiere que se
  cumplan.  Si es posible generar un diseño que contenga todos los
  muebles deseados y cumpla las restricciones puestas, se mostrará tal
  diseño.  Si no es posible, se le mostrará un mensaje indicando el
  porqué del error.
\item[Desplazar un mueble a una posición determinada]El usuario
  selecciona un mueble de la habitación e indica la posición a la que
  quiere desplazarlo.  Si la nueva posición es válida, el mueble se
  desplaza hasta ella.  Si por el contrario se trata de una posición
  inválida, el mueble se mantiene en su posición anterior y se muestra
  un mensaje indicando por qué se ha producido el error.
\item[Guardar diseño] El usuario guarda el estado del diseño actual
  de la habitación en disco de forma permanente.
\item[Exportar diseño a formato imprimible]El usuario exporta el
  diseño en un formato gráfico preparado para su impresión.
\end{description}
\subsubsection{\large{Gestionar catálogo de muebles}}
Engloba las acciones relacionadas con el catálogo de muebles.
\begin{description}
  \item[Añadir mueble]El usuario añade un modelo de mueble al catálogo
    de muebles disponibles. Existen ciertas características que es
    necesario especificar (nombre, tipo, dimensiones, color...) con
    tal de crear un nuevo mueble. También existen características que
    pueden especificarse opcionalmente.
  \item[Modificar mueble]El usuario selecciona un mueble existente en
    el catálogo y modifica algunas de sus características.
  \item[Quitar mueble]El usuario selecciona un mueble existente y lo
    elimina del catálogo.
  \item[Guardar catálogo]El usuario guarda en un fichero externo el
    catálogo de muebles con el que está trabajando en el diseño
    actual. Este incluye los muebles que han sido añadidos o
    modificados, y no incluye los que hayan sido eliminados del
    catálogo durante la ejecución del programa.
  \item[Importar catálogo]El usuario importa desde un fichero externo un
    catálogo determinado de muebles. A partir de este momento, el
    programa trabajará únicamente con los muebles de este catálogo.
\end{description}
\subsubsection{\large{Gestionar catálogo de habitaciones}}
Engloba las acciones relacionadas con el catálogo de habitaciones.
\begin{description}
  \item[Añadir tipo de habitación]El usuario añade un tipo de
    habitación al catálogo de habitaciones disponibles. Existen
    ciertas características que es necesario especificar (tipo,
    dimensiones mínimas, dimensiones máximas...) con tal de crear un
    nuevo mueble. También existen características que pueden
    especificarse opcionalmente.
  \item[Modificar tipo de habitación]El usuario selecciona un tipo de
    habitación existente en el catálogo y modifica algunas de sus
    características.
  \item[Quitar tipo de habitación]El usuario selecciona un tipo de
    habitación existente y lo elimina del catálogo.
  \item[Guardar catálogo]El usuario guarda en un fichero externo el
    catálogo de habitaciones con el que está trabajando en el diseño
    actual. Este incluye los tipos de habitaciones que han sido
    añadidos o modificados, y no incluye los que hayan sido eliminados
    del catálogo durante la ejecución del programa.
  \item[Importar catálogo]El usuario importa desde un fichero externo un
    catálogo determinado de habitaciones. A partir de este momento, el
    programa trabajará únicamente con los tipos de habitación de este
    catálogo.
\end{description}
