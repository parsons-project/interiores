\documentclass[a4paper,11pt,oneside]{article}
\usepackage[utf8]{inputenc}
\usepackage[spanish]{babel}
\usepackage{enumitem}
\usepackage{hyperref}
\hypersetup{
    colorlinks=true,       % false: links enquadrats; true: links en color
    linkcolor=blue,        % color de links interns (peus de pàgina, seccions)
    citecolor=blue,        % color de links a bibliografia (números en [])
    filecolor=magenta,     % color dels links a arxius (externs al pdf)
    urlcolor=blue          % color dels links a urls
}

% Definimos un nuevo comando (\newpar) que básicamente
% añade un espacio vertical de 3mm. Lo hacemos así para
% que haya más espacio que el de por defecto sin modificar
% el parskip (que variaria tambien para titulos y demas)
\newcommand{\newpar}{\vskip 3mm}

\begin{document}
\newcommand{\HRule}{\rule{\linewidth}{0.5mm}}

\begin{titlepage}
\begin{center}

\textsc{\Large Proyecto de Programación}\\[0.5cm]

\HRule \\[0.4cm]
{ \huge \bfseries Diseño de interiores}\\[0.2cm]
\HRule \\[2cm]


\begin{tabular}{c c}
% per ordre alfabètic NOM i username de la FIB
\textsc{Arribas Durango}, Lorenzo & %
\textsc{Espuña Buxo}, Alvaro \\
\texttt{lorenzo.arribas} & %
\texttt{alvaro.espuna} \\[1cm]
\textsc{Mamano Grande}, Nil & %
\textsc{Ramon Jimenez}, Hector \\
\texttt{nil.mamano} & %
\texttt{hector.ramon} \\
\end{tabular}

\vfill

{\large 11 de març de 2013}\\[0.5cm]

\begin{minipage}{0.4\textwidth}
  \begin{flushleft}
    {\large {\bfseries Código de proyecto:} 2.2}
  \end{flushleft}
\end{minipage}
\begin{minipage}{0.4\textwidth}
  \begin{flushright}
    {\large {\bfseries Entrega:} 1.0}
  \end{flushright}
\end{minipage}

\end{center}
\end{titlepage}


\tableofcontents
\clearpage
\section{Análisis de requisitos}
\subsection{Enunciado ampliado}
% De momento es solo una traducción del enunciado original
\subsubsection{Objetivo}
`Diseño de interiores' es un software destinado al diseño de habitaciones
y distribución de su mobiliario. La aplicación es capaz de generar diseños
de habitaciones cumpliendo con una serie de restricciones, permitiendo
al usuario definir cuáles y cómo deben ser estas restricciones.

\subsubsection{Flujo de la aplicación}
En este apartado se explica cuál es el flujo de la aplicación, es decir,
cuáles son las posibilidades del usuario desde que abre el programa hasta que
resuelve su problema. En los siguientes apartados se explica con mucho más
detalle cada uno de los elementos más importantes que conforman el software.

\newpar
Al iniciar la aplicación se le pregunta al usuario si desea:
\begin{description}
  \item[Crear un nuevo diseño]
    En este caso el usuario deberá especificar el tipo de habitación que
    quiere crear y con qué dimensiones.

  \item[Abrir un diseño ya existente]
    El usuario podrá abrir diseños que haya guardado previamente. En este caso
    se cargará la habitación en el mismo estado que cuando se guardó para que
    el usuario pueda seguir trabajando en su diseño.
\end{description}

Una vez el usuario haya creado o abierto una habitación, la aplicación pasará
a mostrarle diversos elementos para facilitarle el diseño de la habitación.
Estos son:

\begin{description}
  \item[Mapa de la habitación]
    Es una representación gráfica de la habitación, dibujando sus accidentes y
    muebles. Su tamaño varía en función de las dimensiones de la habitación.
    Se encontrará en el centro de la aplicación y tendrá un fondo blanco.

  \item[Catálogo de muebles disponibles]
    Menú lateral que contendrá los muebles que se puedan añadir a la
    habitación. Más detalles en el apartado 1.1.5.
  
  \item[Menú de accidentes]
    Menú adyacente al catálogo de muebles disponibles (pestañas?). Mostrará
    los accidentes (detalles en el apartado 1.4) que se puedan añadir a la
    habitación.
\clearpage
  \item[Barra de acciones]
    Es la típica barra situada en la parte superior de la aplicación.
    Permitirá al usuario realizar las siguientes acciones:
    \begin{description}
        \item Crear una nueva habitación
        \item Abrir una habitación
        \item Guardar la habitación
        \item Modificar catálogo de muebles
        \item Modificar restricciones
        \item Acceder a la información Acerca de... de la aplicación.
    \end{description}
\end{description}

El usuario podrá utilizar el catálogo de muebles y el menú de accidentes para
añadir los objetos que desee a su habitación. Para ello solo deberá
seleccionar el mueble o accidente que desee añadir y hacer click en la
posición del mapa donde quiere que se sitúe. (CONDICIONES PARA CONSIDERAR UNA
POSICION COMO CORRECTA)
\newpar
Sin embargo, el principal objetivo de la aplicación es la generación de
diseños de manera automática. Por lo tanto, el usuario podrá especificar en el
catálogo de muebles cuáles deberán incluirse en los diseños que genere la
aplicación. Adicionalmente, el usuario podrá fijar la posición de los muebles
en el mapa. Aquellos muebles que estén fijados no serán redistribuidos por la
aplicación al generar los diseños. Por otra parte, los diseños generados deben
cumplir ciertas restricciones  que el usuario podrá añadir y/o modificar a
través del respectivo menú de la barra de acciones (¿alguna otra idea?).
\newpar

Una vez el usuario haya especificado muebles, accidentes y restricciones podrá
pedir que la aplicación genere o termine su diseño cumpliendo con todas las
restricciones (más información en \ref{restricciones}) que ha establecido
anteriormente. Si no existe ninguna solución que satisfaga las restricciones
se avisará al usuario con un mensaje informativo.

\subsubsection{La habitación}
Una habitación se caracteriza por tener:

\begin{description}
  \item[Dimensiones] Expresadas en centímetros.
  \item[Tipo] Cada tipo de habitación tiene unas restricciones determinadas
  por defecto. Información detallada en el Anexo X.
\end{description}

El usuario podrá añadir a la habitación:

\begin{description}
  \item[Muebles] Utilizando el catálogo de muebles disponibles.
  \item[Accidentes] A través del menú de accidentes.
  \item[Restricciones] Ya sean de la habitación en general o entre muebles
  específicos. Más información en apartado X.
\end{description}

\subsubsection{Los accidentes}
Los accidentes son objetos fijos que se pueden situar en la habitación y que
representan pilares o pilastros. Se utilizan para definir de manera más exacta
la forma de la habitación que se quiere diseñar. Los accidentes no son
redistribuidos cuando la aplicación genera el diseño de la habitación.

\subsubsection{Los muebles}
Los muebles están caracterizados por:

\begin{description}
  \item[Espacio activo] Es el espacio que ocupa el mueble debido a sus
  dimensiones.
  \item[Espacio inactivo] Es el espacio que es necesario para que el mueble
  sea utilizado.
  \item[Color] El color del mueble.
  \item[Material] El material del que está hecho el mueble.
  \item[Precio] Precio orientativo del mueble.
\end{description}

\subsubsection{El cátalogo de muebles}
El catálogo de muebles representa el conjunto total de muebles que están
disponibles en la aplicación. El usuario puede añadir y modificar muebles del
catálogo libremente.

El menú de de muebles disponibles es un menú lateral que contendrá todos los
muebles que puedan añadirse a la habitación en función del tipo de habitación
que se esté diseñando. En primera instancia aparecerán los muebles que
obligatoriamente deban incluirse en el diseño (debe haber una cama en un
dormitorio). A continuación, estarán aquellos muebles que no son de inclusión
obligatoria, pero son comunes en el tipo de habitación (un lavavajillas en la
cocina, por ejemplo). Finalmente aparecerán todos los demás muebles que puedan
incluirse en la habitación. Este menú no contendrá los muebles que NO puedan
ser incluidos en la habitación (es ilógico tener un horno en un baño).

\subsubsection{Las restricciones}
\label{restricciones}

\subsubsection{La generación de diseños}

\clearpage
\subsubsection{Extras}
Adicionalmente, si se dispone del tiempo suficiente se intentarán proporcionar
al usuario las siguientes funcionalidades:

\begin{description}
  \item[Enchufes e interruptores] El usuario podrá posicionar, además de
  muebles y accidentes, enchufes e interruptores en la habitación. Los muebles
  tendrán en cuenta estos elementos para no tapar los interruptores y
  posicionarse relativamente cerca de los enchufes en caso de que necesiten
  electricidad para funcionar.

  \item[Línea de comandos] Una interfaz en línea de comandos donde el usuario
  pueda diseñar su habitación introduciendo comandos. Esto puede ser útil
  sobre todo para debuggear o desarrollar sin necesidad de interfaz gráfica
  y generar diseños de habitaciones a partir de un conjunto de comandos.
\end{description}

\subsection{Diagrama de casos de uso}
% Aqui iria la imagen o lo que sea de los casos de uso con sus descripciones
% del estilo:
% \subsubsection{Caso 1}
% \begin{description}
% \item[Nombre] Caso de Uso 1
% \item[Actores] Jim Carrey
% ...
% \end{description}
\section{Especificación}
\subsection{Diagrama de clases del dominio}
% Más o menos la misma idea que con el diagrama de casos de uso

% Incluir los anexos
\section{Anexo I. Tipos de restricciones disponibles por defecto}
\label{anexo_restricciones}

\section{Anexo II. Tipos de habitaciones disponibles}
\label{anexo_habitaciones}
\vspace{0.5cm}
\begin{adjustwidth}{-1in}{-1in}
  \begin{center}
    \small\addtolength{\tabcolsep}{-2pt}%
    \begin{tabular}{c c c c >{\centering\arraybackslash}m{1.5in} >{\centering\arraybackslash}m{1.5in}}
      \hline \hline
      \textbf{Tipo} & \textbf{Ancho} &
      \textbf{Largo} & \textbf{Altura} &
      \textbf{Superfície mínima} & \textbf{Restricciones}\\
      \hline \hline
  Dormitorio & 180 - 1000 & 240 - 1000 & 240 - 1000 &
  60000 & Tiene que haber o una cama individual o una cama de
  matrimonio. \\ \hline
  Cocina & 120 - 1000 & 240 - 1000 & 220 - 1000 &
  50000 & Tiene que haber una ventana. Tiene que haber una
  vitrocerámica o unos fogones.\\ \hline
  Baño & 100 - 1000 & 210 - 1000 &
  220 - 1000 & 30000 & Tiene que haber una bañera o una ducha.
  Tiene que haber un inodoro.\\ \hline
  Salón & 270 - 1000 & 270 - 1000 & 250 - 1000 &
  72900 & Tiene que haber un sofá. \\ \hline
  Estudio & 180 - 1000 & 200 - 1000 & 240 - 1000 &
  60000 & Tiene que haber una mesa.\\ \hline
  Comedor & 270 - 1000 & 270 - 1000 & 250 - 1000 &
  72900 & Tiene que haber una mesa.\\ \hline
    \end{tabular}
  \end{center}
\end{adjustwidth}

\section{Anexo III. Tipos de modelos de muebles disponibles}
\label{anexo_tipos}

\section{Anexo V. Modelos de muebles disponibles}
\label{anexo_muebles}

\newenvironment{tmobles}[1] % Ens creem un nou environment (begins ends)
               {\subsection{#1} % Referenciem l'argument que rep
                 \vspace{0.25cm}
                 \begin{adjustwidth}{-0.5in}{-1in}
                   \begin{center}
                     \small\addtolength{\tabcolsep}{-2pt}% Aixo es per que hi capiga
                     \begin{tabular}{*{10}{c}}
                       \hline
                       \textbf{Nombre} & \textbf{Precio} & \textbf{Ancho} & \textbf{Largo} &
                       \textbf{Alto} & \textbf{Color} & \textbf{Material} & \textbf{Plazas} &
                       \textbf{Capacidad} \\\hline}%
               {\end{tabular}\end{center}\end{adjustwidth}
                 \vspace{0.5cm}}% Tanquem l'environment

\begin{tmobles}{Camas de matrimonio}
SFORANVIK &  269 & 140 & 205 & 100 & barnizado & pino macizo & 2 & N/A \\
ÄRÖNYN &  229 & 140 & 205 & 40 & barnizado & pino macizo & 2 & N/A \\
SLL &  359 & 140 & 205 & 95 & barnizado & abedul macizo & 2 & N/A \\
KÅMMELLI &  349 & 150 & 190 & 90 & barnizado & haya macizo & 2 & N/A \\
SLÄL &  239 & 160 & 205 & 40 & marrón & aglomerado & 2 & N/A \\
PEKKKA &  239 & 200 & 205 & 55 & marrón & aglomerado & 2 & N/A \\
NYÄLLL &  179 & 140 & 200 & 80 & blanco & aglomerado & 2 & N/A \\
SFLÄTA &  149 & 160 & 205 & 75 & blanco & aglomerado & 2 & N/A \\
KSAYERII &  199 & 150 & 190 & 50 & azul & aglomerado & 2 & N/A \\
IRMMGÅRD &  189 & 140 & 190 & 105 & negro & aglomerado & 2 & N/A \\
\end{tmobles}
\begin{tmobles}{Camas individuales}
TERESJI & 129.99 & 90 & 190 & 50 & barnizado & haya macizo & 1 & N/A\\
BJÖJÄNA & 169.99 & 90 & 205 & 75 & barnizado & pino macizo & 1 & N/A\\
DÄMMH & 169.99 & 80 & 190 & 35 & barnizado & puno macizo & 1 & N/A\\
BJERNNT & 169.99 & 80 & 205 & 50 & barnizado & pino macizo & 1 & N/A\\
RR & 189.99 & 80 & 205 & 40 & barnizado & haya macizo & 1 & N/A\\
EGBJERT & 189.99 & 90 & 200 & 80 & negro & aglomerado & 1 & N/A\\
FJERGÄL & 99.99 & 80 & 200 & 95 & blanco & aglomerado & 1 & N/A\\
RÄDDKKA & 89.99 & 90 & 190 & 85 & marrón & aglomerado & 1 & N/A\\
\end{tmobles}
\begin{tmobles}{Mesillas de noche}
YANVIK & 23.99 & 60 & 40 & 30 & barnizado & haya macizo & N/A & 70\\
ENOK & 36.99 & 40 & 40 & 30 & barnizado & pino macizo & N/A & 50\\
BJÖIKKO & 30.99 & 40 & 40 & 40 & gris & aglomerado & N/A & 55\\
MATYLDD & 79.99 & 60 & 55 & 35 & barnizado & haya macizo & N/A & 95\\
TÅFÄRI & 38.99 & 60 & 30 & 35 & negro & aglomerado & N/A & 65\\
SFKWYPP & 69.99 & 60 & 55 & 40 & barnizado & pino macizo & N/A & 110\\
GEERTJ & 79.99 & 60 & 55 & 40 & barnizado & pino macizo & N/A & 80\\
SNESVSJA & 69.99 & 60 & 40 & 40 & gris & aglomerado & N/A & 85\\
TÄILLANVIK & 49.99 & 45 & 30 & 35 & barnizado & haya macizo & N/A & 40\\
GASVRAYL & 32.99 & 55 & 40 & 35 & barnizado & pino macizo & N/A & 65\\
\end{tmobles}
\begin{tmobles}{Cómodas}
BJYLJÄNNA & 48.99 & 55 & 45 & 110 & marrón & aglomerado & N/A & 270\\
HIPÖLIT &  79.99 & 60 & 45 & 90 & negro & aglomerado & N/A & 235\\
DÄITÖRD &  69.99 & 50 & 50 & 100 & negro & aglomerado & N/A & 230\\
TJEMMA &  129.99 & 60 & 45 & 120 & barnizado & haya maciza & N/A & 305\\
ÄNNGELA &  59.99 & 50 & 40 & 100 & rojo & aglomerado & N/A & 180\\
MYRJÄNA &  89.99 & 70 & 40 & 130 & barnizado & pino macizo & N/A & 370\\
SFYNNIELU &  139.99 & 105 & 45 & 100 & barnizado & haya maciza & N/A & 430\\
ÄN &  69.99 & 60 & 45 & 120 & barnizado & pino macizo & N/A & 335\\
SFYSVKA &  79.99 & 60 & 45 & 140 & barnizado & abedul macizo & N/A & 365\\
NIKKO &  89.99 & 110 & 50 & 95 & azul & aglomerado & N/A & 515\\
ELLIZÄBJETA &  69.99 & 80 & 45 & 120 & azul & aglomerado & N/A & 425\\
\end{tmobles}
\begin{tmobles}{Armarios}
JÅSMINN &  189.99 & 190 & 50 & 200 & marrón & aglomerado & N/A & 1735\\
SÄIYUERD &  89.99 & 110 & 45 & 230 & marrón & aglomerado & N/A & 1075\\
HENRYK &  99.99 & 80 & 50 & 130 & negro & aglomerado & N/A & 520\\
PÅLLI &  169.99 & 160 & 40 & 175 & barnizado & haya maciza & N/A & 1060\\
JÖÄKYMKE &  229.4 & 160 & 50 & 215 & marrón & aglomerado & N/A & 1655\\
TÄS &  59.99 & 85 & 40 & 175 & blanco & aglomerado & N/A & 570\\
EDDANVIK &  149.99 & 185 & 60 & 150 & negro & aglomerado & N/A & 1575\\
SLÖFFYSVA &  139.99 & 160 & 45 & 160 & rojo & aglomerado & N/A & 1080\\
INNGÅLF &  79.99 & 115 & 40 & 150 & negro & aglomerado & N/A & 660\\
ERLÅNTZ &  99.99 & 100 & 35 & 190 & rojo & aglomerado & N/A & 620\\
WILHELLMINN &  69.99 & 60 & 60 & 145 & rojo & aglomerado & N/A & 505\\
BRYT &  149.99 & 100 & 60 & 200 & rojo & aglomerado & N/A & 1150\\
JARÅSJLLÅFA &  239.8 & 195 & 50 & 210 & gris & aglomerado & N/A & 1865\\
ÄFRYSSA &  219.6 & 140 & 60 & 210 & barnizado & pino macizo & N/A & 1730\\
MELISJSJA &  219.8 & 195 & 60 & 180 & blanco & aglomerado & N/A & 2040\\
ÅLLIFERD &  219 & 95 & 40 & 230 & barnizado & haya maciza & N/A & 835\\
YRSVULLA &  89.99 & 65 & 50 & 205 & gris & aglomerado & N/A & 605\\
\end{tmobles}
\begin{tmobles}{Armarios bajos}
ÄRFID &  89.99 & 60 & 60 & 80 & marrón & aglomerado & N/A & 280\\
SLYSIANVIK &  59.99 & 60 & 60 & 95 & rojo & aglomerado & N/A & 320\\
NEFENA &  99.99 & 110 & 45 & 95 & negro & aglomerado & N/A & 450\\
RYMENNA &  89.99 & 90 & 50 & 65 & blanco & aglomerado & N/A & 280\\
RÅLANVIK &  36.99 & 90 & 40 & 80 & azul & aglomerado & N/A & 270\\
SNTÅMMBJI &  49.99 & 80 & 50 & 70 & azul & aglomerado & N/A & 265\\
SVESJNA &  79.99 & 50 & 50 & 65 & rojo & aglomerado & N/A & 155\\
ÄA &  59.99 & 60 & 60 & 95 & gris & aglomerado & N/A & 310\\
SLÄRKE &  89.99 & 90 & 35 & 100 & blanco & aglomerado & N/A & 285\\
SKAANVIK &  89.99 & 110 & 55 & 95 & barnizado & pino macizo & N/A & 560\\
\end{tmobles}
\begin{tmobles}{Armarios altos}
BJÅNYFAÅS &  69.99 & 90 & 30 & 35 & barnizado & pino macizo & N/A & 90\\
ÄDDEBÅYÅLL &  21.99 & 65 & 40 & 50 & rojo & aglomerado & N/A & 130\\
MEENN &  37.99 & 90 & 30 & 45 & negro & aglomerado & N/A & 115\\
\end{tmobles}
\begin{tmobles}{Espejos}
NÖNN &  49.99 & 80 & 5 & 105 & gris & plástico & N/A & N/A\\
TOIRDHELBÄK &  6.99 & 30 & 5 & 70 & blanco & plástico & N/A & N/A\\
SÅFI &  49.99 & 35 & 5 & 70 & blanco & acero & N/A & N/A\\
PI &  109.99 & 100 & 5 & 135 & barnizado & pino macizo & N/A & N/A\\
MÅRT &  59.99 & 90 & 5 & 110 & gris & acero & N/A & N/A\\
BETRYX &  69.99 & 50 & 5 & 80 & gris & acero & N/A & N/A\\
TÅNELLI &  59.99 & 60 & 5 & 90 & gris & acero & N/A & N/A\\
RAGNFÅLD &  31.99 & 90 & 5 & 120 & blanco & plástico & N/A & N/A\\
STJEIND &  89.99 & 80 & 5 & 120 & azul & acero & N/A & N/A\\
\end{tmobles}
\begin{tmobles}{Fregaderos}
TAYKA &  179.99 & 110 & 50 & N/A & gris & acero inoxidable & N/A & N/A\\
SFÄHRI &  119.99 & 80 & 55 & N/A & gris & acero inoxidable & N/A & N/A\\
TJEÅDDÅRD &  89.99 & 115 & 50 & N/A & gris & acero inoxidable & N/A & N/A\\
KÖMGÄL &  189.99 & 100 & 60 & N/A & gris & acero inoxidable & N/A & N/A\\
HERMÄNDÖRD &  219 & 120 & 60 & N/A & gris & acero inoxidable & N/A & N/A\\
\end{tmobles}
\begin{tmobles}{Hornos}
NÖHN &  259 & 60 & 55 & 60 & gris & acero inoxidable & N/A & N/A\\
SVILFFIO &  239 & 60 & 55 & 60 & gris & acero inoxidable & N/A & N/A\\
SVUREK &  209 & 60 & 55 & 60 & gris & acero inoxidable & N/A & N/A\\
TUYLLA &  229 & 60 & 55 & 60 & gris & acero inoxidable & N/A & N/A\\
\end{tmobles}
\begin{tmobles}{Microondas}
SEIJA &  229 & 60 & 35 & 35 & gris & acero inoxidable & N/A & N/A\\
ÄLEKSVÅNTÖRD &  99.99 & 60 & 35 & 35 & gris & acero inoxidable & N/A & N/A\\
MÅXIMMILLI &  99.99 & 60 & 35 & 35 & gris & acero inoxidable & N/A & N/A\\
FJYÅNNDÖRD &  99.99 & 60 & 35 & 35 & gris & acero inoxidable & N/A & N/A\\
\end{tmobles}
\begin{tmobles}{Vitrocerámicas}
HOA &  279 & 60 & 50 & N/A & negro & vidrio & N/A & N/A\\
ÄNNTÖNYN &  309 & 60 & 50 & N/A & negro & vidrio & N/A & N/A\\
NYRU &  199.99 & 60 & 50 & N/A & negro & vidrio & N/A & N/A\\
SÅMMYEL &  419 & 60 & 50 & N/A & gris & vidrio & N/A & N/A\\
GÖRDDANVIK &  419 & 60 & 50 & N/A & negro & vidrio & N/A & N/A\\
\end{tmobles}
\begin{tmobles}{Fogones}
KRSTO &  199.99 & 60 & 50 & N/A & gris & acero inoxidable & N/A & N/A\\
ÄLBINN &  219 & 75 & 50 & N/A & gris & acero inoxidable & N/A & N/A\\
SLÅTARVIK &  149.99 & 60 & 50 & N/A & gris & acero inoxidable & N/A & N/A\\
GLÖRYNTA &  339 & 70 & 50 & N/A & gris & acero inoxidable & N/A & N/A\\
SLYUK &  309 & 75 & 50 & N/A & gris & acero inoxidable & N/A & N/A\\
\end{tmobles}
\begin{tmobles}{Extractores}
MYRKE &  249 & 60 & 50 & 110 & gris & acero & N/A & N/A\\
MÖDDESJT &  389 & 60 & 50 & 95 & gris & acero & N/A & N/A\\
MÅRI &  359 & 90 & 50 & 100 & gris & acero & N/A & N/A\\
MÄLLÄKKII &  229 & 60 & 45 & 85 & gris & acero & N/A & N/A\\
JELYSVÄFETA &  209 & 90 & 50 & 110 & gris & acero & N/A & N/A\\
SKÅNRÅD &  279 & 90 & 50 & 95 & gris & acero & N/A & N/A\\
\end{tmobles}
\begin{tmobles}{Neveras}
ÄRZU &  389 & 60 & 60 & 180 & gris & acero inoxidable & N/A & 445\\
ÄDDELLA &  269 & 60 & 60 & 185 & blanco & acero inoxidable & N/A & 510\\
HILLTRÅD &  369 & 60 & 60 & 180 & gris & acero inoxidable & N/A & 475\\
SNONNTLL &  279 & 60 & 60 & 185 & gris & acero inoxidable & N/A & 460\\
KYASI &  479 & 60 & 60 & 185 & gris & acero inoxidable & N/A & 455\\
MYÅDDRAG &  549 & 60 & 60 & 180 & blanco & acero inoxidable & N/A & 455\\
DEYRDDR &  499 & 60 & 60 & 185 & blanco & acero inoxidable & N/A & 475\\
SVLLÅDDIMKE &  459 & 60 & 60 & 180 & blanco & acero inoxidable & N/A & 450\\
ELLYF &  419 & 60 & 60 & 185 & blanco & acero inoxidable & N/A & 465\\
\end{tmobles}
\begin{tmobles}{Lavavajillas}
PÅSJKÄL &  249 & 60 & 55 & 75 & gris & acero inoxidable & N/A & N/A\\
RÅK &  229 & 60 & 55 & 80 & blanco & acero inoxidable & N/A & N/A\\
TEREZA &  249 & 60 & 60 & 75 & gris & acero inoxidable & N/A & N/A\\
EDDMYND &  259 & 60 & 55 & 80 & blanco & acero inoxidable & N/A & N/A\\
SERPYL &  219 & 60 & 55 & 75 & blanco & acero inoxidable & N/A & N/A\\
\end{tmobles}
\begin{tmobles}{Lavadoras}
SLIUBJÅMIR &  219 & 60 & 55 & 85 & gris & acero inoxidable & N/A & N/A\\
REIND &  219 & 60 & 60 & 75 & negro & acero inoxidable & N/A & N/A\\
HENRIKA &  279 & 60 & 55 & 85 & gris & acero inoxidable & N/A & N/A\\
SFRYSTII &  229 & 60 & 60 & 75 & gris & acero inoxidable & N/A & N/A\\
WYLHELLMKE &  389 & 60 & 55 & 80 & blanco & acero inoxidable & N/A & N/A\\
\end{tmobles}
\begin{tmobles}{Estantes}
KTYRAD & 37.99 & 70 & 10 & N/A & barnizado & haya maciza & N/A & N/A\\
BJERTRÅMKE & 20.99 & 70 & 16 & N/A & barnizado & pino macizo & N/A & N/A\\
REBJEKKSSA & 28.99 & 100 & 10 & N/A & barnizado & pino macizo & N/A & N/A\\
ESVTERKE & 39.99 & 75 & 12 & N/A & barnizado & haya maciza & N/A & N/A\\
ÄNA & 19.99 & 65 & 10 & N/A & marrón & aglomerado & N/A & N/A\\
DRÅGO & 24.99 & 95 & 12 & N/A & gris & aglomerado & N/A & N/A\\
RENNÅRD & 28.99 & 90 & 17 & N/A & gris & aglomerado & N/A & N/A\\
EMMA & 27.99 & 65 & 13 & N/A & barnizado & pino macizo & N/A & N/A\\
RETO & 30.99 & 110 & 13 & N/A & blanco & aglomerado & N/A & N/A\\
SEBASTIANVIK & 17.99 & 70 & 10 & N/A & blanco & aglomerado & N/A & N/A\\
DIDA & 24.99 & 75 & 15 & N/A & barnizado & pino macizo & N/A & N/A\\
SFÅHARINN & 20.99 & 75 & 18 & N/A & gris & aglomerado & N/A & N/A\\
ESVMKE & 59.99 & 105 & 14 & N/A & barnizado & abedul macizo & N/A & N/A\\
PÄITRÅNNELLA & 31.99 & 115 & 11 & N/A & barnizado & pino macizo & N/A & N/A\\
DOMMINYKKA & 34.99 & 50 & 22 & N/A & barnizado & abedul macizo & N/A & N/A\\
\end{tmobles}
\begin{tmobles}{Papeleras}
DÄIDDERIKKH & 8.99 & 40 & 40 & 55 & rojo & acero & N/A & N/A\\
SFYDDÄIGUU & 13.99 & 40 & 40 & 50 & naranja & acero & N/A & N/A\\
TÄNELLI & 12.99 & 35 & 35 & 40 & naranja & acero & N/A & N/A\\
ELYZA & 15.99 & 30 & 30 & 50 & blanco & acero & N/A & N/A\\
DYRKEN & 7.99 & 30 & 30 & 40 & gris & plástico & N/A & N/A\\
\end{tmobles}
\begin{tmobles}{Bidés}
ÄLLEKKSVEI & 89.99 & 40 & 60 & 35 & gris & porcelana & N/A & N/A\\
SVYLMMA & 159.99 & 40 & 60 & 35 & blanco & porcelana & N/A & N/A\\
GREGÅRD & 189.99 & 40 & 60 & 35 & blanco & porcelana & N/A & N/A\\
RAMÖNNA & 149.99 & 40 & 60 & 40 & blanco & porcelana & N/A & N/A\\
DEKEBJÅL & 209 & 40 & 60 & 35 & gris & porcelana & N/A & N/A\\
\end{tmobles}
\begin{tmobles}{Lavamanos}
SKLLÅDDIYSJZ & 37.99 & 55 & 45 & 90 & blanco & porcelana & N/A & N/A\\
SFYNYEYKKEN & 49.99 & 50 & 40 & 100 & blanco & porcelana & N/A & N/A\\
HELLENA & 49.99 & 45 & 35 & 90 & blanco & porcelana & N/A & N/A\\
MIKEL & 79.99 & 60 & 55 & 95 & blanco & porcelana & N/A & N/A\\
SADDB & 39.99 & 55 & 50 & 90 & azul & porcelana & N/A & N/A\\
SFÄRLLOTT & 39.99 & 60 & 50 & 100 & gris & porcelana & N/A & N/A\\
BRYGYD & 49.99 & 55 & 50 & 100 & blanco & porcelana & N/A & N/A\\
ETNNA & 59.99 & 60 & 55 & 90 & gris & porcelana & N/A & N/A\\
\end{tmobles}
\begin{tmobles}{Inodoros}
SVÖJTEKH & 219 & 55 & 70 & 40 & negro & porcelana & N/A & N/A\\
SNYKKOLLÅS & 299 & 40 & 75 & 40 & blanco & porcelana & N/A & N/A\\
KÅRLLA & 139.99 & 55 & 75 & 40 & blanco & porcelana & N/A & N/A\\
ÄYMMO & 299 & 40 & 75 & 40 & blanco & porcelana & N/A & N/A\\
SVID & 179.99 & 55 & 75 & 40 & blanco & porcelana & N/A & N/A\\
SLUUK & 109.99 & 40 & 75 & 40 & blanco & porcelana & N/A & N/A\\
HEINNRYKH & 159.99 & 55 & 75 & 40 & blanco & porcelana & N/A & N/A\\
BÅRBÄRKE & 89.99 & 55 & 75 & 40 & negro & porcelana & N/A & N/A\\
RODDIA & 109.99 & 40 & 75 & 40 & blanco & porcelana & N/A & N/A\\
\end{tmobles}
\begin{tmobles}{Bañeras}
SLÅRS & 129 & 70 & 160 & 50 & blanco & porcelana & N/A & N/A\\
META & 329 & 70 & 190 & 45 & negro & porcelana & N/A & N/A\\
RYYD & 259 & 70 & 175 & 45 & blanco & porcelana & N/A & N/A\\
DÅGNNII & 149 & 70 & 170 & 40 & blanco & porcelana & N/A & N/A\\
MIKKITA & 319 & 75 & 170 & 45 & gris & porcelana & N/A & N/A\\
MYTYSSA & 239 & 80 & 170 & 45 & negro & porcelana & N/A & N/A\\
SIGN & 319 & 70 & 175 & 50 & blanco & porcelana & N/A & N/A\\
STJIANVIK & 239 & 80 & 170 & 45 & blanco & porcelana & N/A & N/A\\
\end{tmobles}
\begin{tmobles}{Duchas}
EMMILIIA & 139.99 & 90 & 120 & N/A & blanco & porcelana & N/A & N/A\\
KIYRKE & 119.99 & 70 & 90 & N/A & blanco & porcelana & N/A & N/A\\
SLEA & 109.99 & 70 & 100 & N/A & gris & porcelana & N/A & N/A\\
MARGARETA & 99.99 & 70 & 100 & N/A & blanco & porcelana & N/A & N/A\\
SÄLÅM & 69.99 & 70 & 70 & N/A & negro & porcelana & N/A & N/A\\
STJELLÅRKE & 129.99 & 80 & 100 & N/A & blanco & porcelana & N/A & N/A\\
ÄDDELA & 79.99 & 70 & 70 & N/A & negro & porcelana & N/A & N/A\\
FJINYANVIK & 109.99 & 70 & 100 & N/A & blanco & porcelana & N/A & N/A\\
\end{tmobles}
\begin{tmobles}{Secadoras}
KELLÅGH & 239 & 60 & 55 & 75 & gris & acero inoxidable & N/A & N/A\\
WYKTÅRD & 299 & 60 & 55 & 80 & blanco & acero inoxidable & N/A & N/A\\
SLUGHÄYDDH & 389 & 60 & 60 & 75 & negro & acero inoxidable & N/A & N/A\\
SKÅTJA & 269 & 60 & 55 & 80 & blanco & acero inoxidable & N/A & N/A\\
\end{tmobles}
\begin{tmobles}{Mesas}
JENGO & 59.99 & 90 & 140 & 90 & barnizado & pino macizo & 5 & N/A\\
ÄTA & 59.99 & 70 & 110 & 85 & marrón & plástico & 3 & N/A\\
EMIL & 69.99 & 115 & 190 & 105 & azul & aglomerado & 9 & N/A\\
KÖRINN & 149.99 & 145 & 160 & 105 & barnizado & haya maciza & 9 & N/A\\
HÅNNÅRÅTA & 119.99 & 75 & 130 & 90 & gris & aglomerado & 4 & N/A\\
SVYNNKKENNC & 119.99 & 140 & 160 & 90 & negro & aglomerado & 9 & N/A\\
ÅILI & 169.99 & 120 & 230 & 90 & marrón & aglomerado & 11 & N/A\\
ÄAPPELI & 79.99 & 105 & 155 & 85 & azul & aglomerado & 6 & N/A\\
WÄIB & 109.99 & 145 & 195 & 110 & azul & vidrio & 11 & N/A\\
ÄGÄTA & 129.99 & 120 & 185 & 110 & barnizado & pino macizo & 9 & N/A\\
RÖBJERT & 169.99 & 80 & 135 & 90 & barnizado & pino macizo & 4 & N/A\\
FJLÅRYS & 169.99 & 90 & 160 & 90 & barnizado & pino macizo & 6 & N/A\\
WÅLDEMARVIK & 119.99 & 115 & 160 & 105 & barnizado & pino macizo & 7 & N/A\\
JERKÖRD & 69.99 & 120 & 180 & 90 & gris & plástico & 9 & N/A\\
SNIKÖLLA & 129.99 & 95 & 185 & 100 & azul & aglomerado & 7 & N/A\\
BJERGLLYÅT & 36.99 & 115 & 130 & 105 & verde & plástico & 6 & N/A\\
SVULLIJANA & 69.99 & 135 & 195 & 90 & blanco & vidrio & 10 & N/A\\
KONÅRD & 139.99 & 90 & 155 & 100 & azul & vidrio & 5 & N/A\\
SKRISTYINN & 99.99 & 100 & 180 & 80 & rojo & aglomerado & 7 & N/A\\
PÅFLLA & 119.99 & 120 & 135 & 85 & barnizado & pino macizo & 6 & N/A\\
ÄNYKKA & 39.99 & 70 & 110 & 80 & blanco & aglomerado & 3 & N/A\\
ÄNNTTÖNNI & 49.99 & 55 & 70 & 80 & gris & aglomerado & 1 & N/A\\
\end{tmobles}
\begin{tmobles}{Sillas}
SÅMMHAYRLL & 14.99 & 45 & 45 & 110 & negro & plástico & 1 & N/A\\
MYTJA & 12.99 & 45 & 45 & 115 & blanco & plástico & 1 & N/A\\
MÄRGARETA & 27.99 & 50 & 50 & 90 & gris & aglomerado & 1 & N/A\\
ILLSA & 16.99 & 45 & 45 & 105 & blanco & plástico & 1 & N/A\\
EMÖRD & 10.99 & 45 & 45 & 105 & negro & plástico & 1 & N/A\\
NADDA & 22.99 & 50 & 50 & 95 & barnizado & pino macizo & 1 & N/A\\
MOIRRI & 20.99 & 45 & 45 & 105 & blanco & aglomerado & 1 & N/A\\
HEDDFFIG & 34.99 & 50 & 50 & 105 & barnizado & pino macizo & 1 & N/A\\
SÅBIENRA & 18.99 & 50 & 50 & 125 & amarillo & plástico & 1 & N/A\\
ÅLIMPI & 27.99 & 50 & 50 & 110 & negro & aglomerado & 1 & N/A\\
ÄMÅLI & 29.99 & 50 & 50 & 95 & gris & aglomerado & 1 & N/A\\
\end{tmobles}
\begin{tmobles}{Sillones}
JÖSUNN & 38.99 & 60 & 60 & 90 & azul & aglomerado & 1 & N/A\\
SKÄTJA & 59.99 & 70 & 70 & 120 & marrón & aglomerado & 1 & N/A\\
ÄRKKÅITZ & 59.99 & 70 & 70 & 115 & azul & aglomerado & 1 & N/A\\
\end{tmobles}
\begin{tmobles}{Trozos de mármol}
MIHÄELLA & 17.99 & 35 & 60 & 90 & blanco & marmol & N/A & N/A\\
SVYRGYLYU & 20.99 & 40 & 60 & 90 & blanco & marmol & N/A & N/A\\
GYNNTRAMKE & 22.99 & 45 & 60 & 90 & blanco & marmol & N/A & N/A\\
SEPPPPO & 25.99 & 50 & 60 & 90 & blanco & marmol & N/A & N/A\\
SYGRUND & 27.99 & 55 & 60 & 90 & blanco & marmol & N/A & N/A\\
WIBJO & 30.99 & 60 & 60 & 90 & blanco & marmol & N/A & N/A\\
EDDFÄRD & 49.99 & 80 & 60 & 90 & blanco & marmol & N/A & N/A\\
HELMMA & 49.99 & 90 & 60 & 90 & blanco & marmol & N/A & N/A\\
RYFINN & 59.99 & 100 & 60 & 90 & blanco & marmol & N/A & N/A\\
\end{tmobles}
\begin{tmobles}{Escurridores}
MÄRTIND & 59.99 & 60 & 60 & 60 & gris & acero inoxidable & N/A & N/A\\
MAAYKKEN & 59.99 & 60 & 60 & 60 & gris & acero inoxidable & N/A & N/A\\
EDDYT & 49.99 & 60 & 60 & 60 & gris & acero inoxidable & N/A & N/A\\
GERTRÅD & 69.99 & 60 & 60 & 60 & gris & acero inoxidable & N/A & N/A\\
\end{tmobles}
\begin{tmobles}{Sofás}
SYSVÅNNDÖRD & 329.99 & 155 & 60 & 120 & negro & acero & 2 & N/A\\
SJEF & 599.60 & 180 & 60 & 120 & blanco & haya maciza & 3 & N/A\\
SLEONNÖR & 79.99 & 130 & 60 & 90 & rojo & plástico & 2 & N/A\\
MÄRGÅREETA & 199.99 & 180 & 60 & 115 & verde & plástico & 3 & N/A\\
ÄNANVIK & 359.99 & 155 & 60 & 95 & azul & aglomerado & 2 & N/A\\
KÖMGANVIK & 299.99 & 155 & 60 & 100 & blanco & pino macizo & 2 & N/A\\
SVEDDRÄNA & 329.99 & 210 & 60 & 100 & rojo & pino macizo & 3 & N/A\\
MYRÖSVLÅSVA & 269.99 & 150 & 60 & 100 & marrón & aglomerado & 2 & N/A\\
DÄLLIBORKKA & 209.99 & 175 & 60 & 120 & amarillo & plástico & 3 & N/A\\
ÄISJU & 189.99 & 180 & 60 & 115 & gris & acero & 3 & N/A\\
KRYSJTINÅN & 309.99 & 220 & 60 & 110 & rojo & plástico & 3 & N/A\\
TJYLO & 229.10 & 205 & 55 & 95 & marrón & aglomerado & 3 & N/A\\
RÅZA & 239.30 & 180 & 60 & 100 & marrón & acero & 3 & N/A\\
SVÖNYN & 189.99 & 210 & 60 & 110 & gris & acero & 3 & N/A\\
DÄFFÅRKKA & 149.99 & 145 & 60 & 120 & rojo & aglomerado & 2 & N/A\\
MEYNTJ & 369.99 & 145 & 60 & 105 & marrón & aglomerado & 2 & N/A\\
SLYDDIIA & 469.40 & 210 & 60 & 110 & rojo & haya maciza & 3 & N/A\\
EMILI & 179.99 & 125 & 60 & 120 & blanco & pino macizo & 2 & N/A\\
JOZEF & 229.20 & 130 & 55 & 95 & blanco & pino macizo & 2 & N/A\\
SFYOMA & 199.99 & 210 & 60 & 110 & azul & pino macizo & 3 & N/A\\
\end{tmobles}
\begin{tmobles}{Estanterías}
GÅRÄYL & 349 & 250 & 40 & 215 & blanco & aglomerado & N/A & N/A\\
SVYSVEKKA & 159 & 210 & 30 & 135 & barnizado & pino macizo & N/A & N/A\\
IBRÄHYMMA & 329 & 125 & 30 & 195 & barnizado & pino macizo & N/A & N/A\\
SVÄLERII & 249 & 135 & 30 & 220 & rojo & aglomerado & N/A & N/A\\
BERYT & 359 & 105 & 40 & 245 & marrón & aglomerado & N/A & N/A\\
ÄUGUSTYS & 279 & 165 & 35 & 110 & gris & aglomerado & N/A & N/A\\
ÄNNITA & 349 & 110 & 50 & 170 & barnizado & pino macizo & N/A & N/A\\
IDDA & 269 & 145 & 24 & 195 & barnizado & pino macizo & N/A & N/A\\
WOLF & 199.99 & 40 & 55 & 250 & rojo & aglomerado & N/A & N/A\\
MÅRJÄNNI & 339 & 235 & 30 & 110 & negro & aglomerado & N/A & N/A\\
POLINN & 239 & 180 & 30 & 210 & verde & plástico & N/A & N/A\\
ÄNNELI & 49.99 & 55 & 30 & 165 & blanco & plástico & N/A & N/A\\
MARTINN & 199.99 & 135 & 25 & 130 & azul & aglomerado & N/A & N/A\\
\end{tmobles}
\begin{tmobles}{Mesas bajas}
SVERÖNYKA & 59.99 & 50 & 40 & 35 & marrón & aglomerado & N/A & N/A\\
SSANIU & 79.99 & 55 & 30 & 40 & rojo & aglomerado & N/A & N/A\\
BJO & 49.99 & 50 & 30 & 35 & negro & aglomerado & N/A & N/A\\
TINNEK & 59.99 & 40 & 30 & 35 & marrón & aglomerado & N/A & N/A\\
GERTY & 49.99 & 60 & 30 & 30 & barnizado & pino macizo & N/A & N/A\\
TIGO & 26.99 & 40 & 35 & 40 & negro & plástico & N/A & N/A\\
\end{tmobles}
\begin{tmobles}{Percheros}
SKÅRD & 79.99 & 45 & 45 & 220 & barnizado & pino macizo & N/A & N/A\\
DÄNYSSA & 59.99 & 60 & 60 & 195 & negro & aglomerado & N/A & N/A\\
MYLÄNNA & 89.99 & 40 & 40 & 200 & barnizado & pino macizo & N/A & N/A\\
SEKKNALLL & 99.99 & 50 & 50 & 220 & barnizado & pino macizo & N/A & N/A\\
RYDDÖLLF & 79.99 & 60 & 60 & 195 & marrón & aglomerado & N/A & N/A\\
\end{tmobles}
\begin{tmobles}{Televisores}
HELG & 149.99 & 80 & 30 & 75 & negro & plastico & N/A & N/A \\
SLÅTARVIK & 299 & 85 & 10 & 75 & gris & plastico & N/A & N/A \\
GÖRDDANVIK & 129.99 & 55 & 45 & 50 & negro & plastico & N/A & N/A \\
WITOLLD & 119.99 & 50 & 40 & 45 & negro & plastico & N/A & N/A \\
SFREDDRYK & 319 & 80 & 5 & 70 & negro & plastico & N/A & N/A \\
SNKEKKI & 99.99 & 55 & 35 & 45 & negro & plastico & N/A & N/A \\
TEÖDDÅRD & 129.99 & 85 & 45 & 75 & negro & plastico & N/A & N/A \\
ÄRISJA & 99.99 & 75 & 30 & 60 & gris & plastico & N/A & N/A \\
\end{tmobles}
\begin{tmobles}{Bufetes}
GÅRANVIK & 59.99 & 35 & 45 & 75 & azul & aglomerado & N/A & 90 \\
ÄLLF & 69.99 & 55 & 65 & 70 & marron & aglomerado & N/A & 190 \\
TOMMA & 89.99 & 40 & 50 & 65 & barnizado & abedul macizo & N/A & 95 \\
BÅRBÄRKE & 99.99 & 65 & 65 & 60 & negro & aglomerado & N/A & 200 \\
BJÖRISJLLÅFA & 59.99 & 45 & 55 & 90 & blanco & aglomerado & N/A & 160 \\
SVÅANNDÖRD & 39.99 & 40 & 40 & 70 & gris & aglomerado & N/A & 80 \\
DÄRINN & 24.99 & 50 & 50 & 70 & azul & aglomerado & N/A & 140 \\
SLÖFFYSVA & 109.99 & 75 & 75 & 90 & blanco & aglomerado & N/A & 365 \\
SVÖAKKYMKE & 35.99 & 60 & 65 & 80 & blanco & aglomerado & N/A & 245 \\
SNEDDJELLJKO & 69.99 & 70 & 80 & 100 & barnizado & pino macizo & N/A & 410 \\
MIRKE & 49.99 & 70 & 80 & 80 & negro & aglomerado & N/A & 350 \\
MÄLLÄKKII & 49.99 & 55 & 65 & 100 & rojo & aglomerado & N/A & 285 \\
SÅFFA & 69.99 & 30 & 35 & 90 & negro & aglomerado & N/A & 70 \\
\end{tmobles}



\end{document}
